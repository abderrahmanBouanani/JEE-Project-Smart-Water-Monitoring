%%%%%%%%%%%%%%%%%%%%%%%%%%%%%%%%%%%%%%%%%%%%%%%%%%%%%%%%%%%%%%%%%%%%%%%%%%%%%%%%%%%%
%%----------------------------------------------------------------------------------
% DO NOT Change this is the required setting A4 page, 11pt, onside print, book style
%%----------------------------------------------------------------------------------
\documentclass[a4paper,11pt,oneside]{book} 
\usepackage{main_report} % DO NOT REMOVE THIS LINE. 
\usepackage[T1]{fontenc}
\usepackage[utf8]{inputenc}
\usepackage{graphicx}
\usepackage{geometry}
\geometry{margin=2.5cm}
\usepackage{etoolbox}
\usepackage{tcolorbox}
\usepackage[french]{babel}
\usepackage{listings} 

% Paramètres du tcolorbox pour obtenir des coins arrondis
\tcbset{
  title style/.style={font=\Large\bfseries},
  colback=white,
  colframe=black,
  boxrule=0.7pt,
  arc=5mm,
  left=20pt,
  right=20pt,
  top=20pt,
  bottom=20pt
}

% Redéfinition de \today pour avoir le jour et le mois en français
\renewcommand{\today}{%
  \number\day~%
  \ifcase\month\or
    janvier\or février\or mars\or avril\or mai\or juin\or
    juillet\or août\or septembre\or octobre\or novembre\or décembre
  \fi
  \space \number\year
}


%%%%%%%%%%%%%%%%%%%%%%%%%%%%%%%%%%%%%%%%%%%%%%%%%%%%%%%%%%%%%%%%%%%%%%%%%%%%%%%%%%%%


%%%%%%%%%%%%%%%%%%%%%%%%%%%%%%%%%%%%%%%%%%%%%%%%%%%%%%%%%%%%%%%%%%%%%%%%%%%%%%%%%%%%
\begin{document}
    \captionsetup[figure]{margin=1.5cm,font=small,name={Figure},labelsep=colon}
    \captionsetup[table]{margin=1.5cm,font=small,name={Table},labelsep=colon}
    
    \frontmatter
    
    %%%%%%%%%%%%%%%%%%%%%%%%%%%%%%%%%%%%%%%%%%%%%%%%%%%%%%%%%%%%%%%%%%%%%%%%%%%%%%%%
    \newgeometry{top=1.5cm, bottom=1.5cm, left=2cm, right=2cm} 

\begin{titlepage}
  \thispagestyle{empty} % Pas de numéro de page ni d'en-tête
  
  \begin{center}
    \begin{tcolorbox}[height=\textheight]
      \begin{center}
        
        \vspace{0.75cm}

        % Logo de l'établissement
        \includegraphics[width=8cm]{figures/ensaa-logo.png}\\[1cm]

        \vspace{2cm}

        {
          \textbf{Ecole Nationale Supérieure des Sciences Appliquées Agadir - ENSAA}\\[0.5cm]
          \textbf{Université Ibn Zohr - UIZ}\\[0.5cm]
        }
        
        % Ligne de séparation
        \rule{\linewidth}{0.8pt}\\[1.0cm]
        
        % Titre du rapport
        {\huge
          \textbf{Rapport du projet de fin du module JEE}
        }\\[0.5cm]
        {\Large
          \textit{Plateforme de Supervision de la Consommation d'Eau}
        }\\[1.2cm]
        
        % Ligne de séparation
        \rule{\linewidth}{0.8pt}\\[0.8cm]
        
        
        % Informations sur l'auteur
        
        {\large
          \textbf{Réalisé par :}\\[0.3cm]
          \begin{tabular}{l}
            Abderrahman BOUANANI \\
            Abou Kekeli Efrayim
          \end{tabular}
        }\\[0.8cm]
        
        % Informations sur l'encadrant
        {\large
          \textbf{Encadré par :}\\[0.3cm]
          \begin{tabular}{l}
            Pr. BENIDER
          \end{tabular}
        }\\[0.8cm]
        

        \vspace{4cm}
        % Détails du diplôme et année universitaire
        {\large
          Filière: DLA2 - 2ème Année Développement Logiciel et Applicatif\\[0.5cm]
          Année Universitaire : 2025--2026
        }\\[1.0cm]
        
        % Date (facultatif)
        % \today
      \end{center}
    \end{tcolorbox}
  \end{center}
\end{titlepage}
\restoregeometry
    
    
    % -------------------------------------------------------------------
    % Declaration
    % -------------------------------------------------------------------
    \newpage
    \thispagestyle{empty}
    \chapter*{Déclaration}

Je soussigné(e), \textbf{Abderrahman Bouanani}, de l’ENSA Agadir, déclare que le présent rapport, y compris l’ensemble des textes, figures, tableaux, extraits de code et illustrations, constitue le fruit de mon travail personnel. Les sources externes utilisées ont fait l’objet de citations et références explicites. Je reconnais que tout manquement à cet engagement relèverait du plagiat, considéré comme une infraction académique passible de sanctions.

\medskip

\noindent
J’autorise la mise à disposition d’un exemplaire de ce rapport à des fins pédagogiques, 
au bénéfice des futurs étudiants et chercheurs, et j’accepte qu’il soit consulté dans 
l’intérêt général de l’Enseignement Supérieur et de la Recherche.

\vspace{1cm}

\begin{flushright}
\textbf{Abderrahman Bouanani}\\
\today
\end{flushright}


    
    % -------------------------------------------------------------------
	% Acknowledgement
	% -------------------------------------------------------------------
   
    \chapter*{Remerciements}
\addcontentsline{toc}{chapter}{Remerciements}

Nous tenons à exprimer notre profonde gratitude à notre encadrant, le Professeur BENIDER, pour son accompagnement précieux, sa disponibilité et ses conseils avisés tout au long de la réalisation de ce projet. Son expertise et son soutien ont été des atouts majeurs pour le bon déroulement de notre travail.

Nos remerciements s’adressent également à l’ensemble du corps professoral de l’ENSA Agadir pour la qualité de la formation dispensée.

Enfin, nous n’oublions pas nos familles et nos amis pour leur soutien inconditionnel et leurs encouragements constants.
   

     % -------------------------------------------------------------------
    % Abstract and Acknowledgement
    % -------------------------------------------------------------------
    
    \chapter*{Abstract}
\addcontentsline{toc}{chapter}{Abstract}

This document provides a template for a project report along with guidelines on how to write a report. It also includes several useful examples to help you become accustomed to \LaTeX. The number and titles of the chapters may vary depending on the type of project and individual preferences. The section titles presented here are only illustrative and should be adapted as necessary. Likewise, the number of sections within each chapter remains flexible. This template may or may not suit your project, so it is advisable to discuss the structure of your report with your supervisor.

\medskip

\noindent\textbf{Guidelines for writing an abstract:}  
The abstract is a summary of the report, presented in a single paragraph and not exceeding 250 words. It must be \textit{self-contained} and should not refer to other sections, figures, tables, equations, or references. In general, an abstract comprises four key elements:
\begin{enumerate}
    \item \textbf{Introduction} (background and purpose of the project);
    \item \textbf{Methods} (experiments, techniques, or implementation);
    \item \textbf{Results} (main conclusions obtained, along with their significance);
    \item \textbf{Conclusions} (implications for the field of study).
\end{enumerate}
The distribution of these four parts should reflect their relative importance within the body of the report. The abstract begins with a few sentences describing the general theme and main objective of the project, then presents the targeted problem. This is followed by a brief description of the methodology employed, before summarizing the results obtained and their meaning. Finally, the conclusion highlights the project’s major contributions and its impact on the field.

\vspace{1cm}
\noindent
\textbf{Keywords:} Up to five keywords or key phrases, separated by commas

\vfill

\noindent
\textbf{Report's Total Word Count:} The word count must be stated after the abstract. 
The report should contain at least 10,000 words and at most 15,000 words (counting from Chapter 1 
through the end of the Conclusions chapter, but excluding the list of references, appendices, abstract, 
and text contained in figures, tables, listings, and captions), roughly 40 to 50 pages.

\medskip

\noindent
\textbf{The project's source code must be uploaded to GitLab. The GitLab link should be included here, following the word count.}

\medskip

\noindent
The report (preferably in PDF format) must be submitted via the "Ecampus" platform before the deadline. 
If a student has resit examinations for any taught modules, the submission deadline for the dissertation 
will be postponed by two weeks from the original deadline.

    \chapter*{Résumé}
\addcontentsline{toc}{chapter}{Résumé}

La rareté de l'eau et la gestion inefficace des ressources posent des défis majeurs pour le développement urbain durable. Ce projet présente Smart Water Monitoring, une plateforme web complète pour la gestion intelligente de la consommation d'eau utilisant la technologie Internet des Objets (IoT) et Java Enterprise Edition (JEE). Le système répond au besoin critique de surveillance en temps réel, d'analyse automatisée des données et de gestion proactive des alertes pour optimiser l'utilisation des ressources hydriques dans les environnements résidentiels. La plateforme implémente une architecture JEE multi-tiers suivant le patron Modèle-Vue-Contrôleur, utilisant l'API Jakarta Servlet pour le traitement des requêtes, Hibernate ORM pour la persistance des données, et MySQL 8 pour le stockage. Les fonctionnalités principales incluent la collecte de données de capteurs IoT en temps réel via des API REST, l'agrégation quotidienne automatique de la consommation par tâches planifiées, la génération intelligente d'alertes basée sur des seuils configurables et des motifs temporels, l'analyse statistique complète, et le contrôle d'accès basé sur les rôles pour administrateurs et citoyens. Les mesures de sécurité intègrent le hachage BCrypt des mots de passe avec salage à 12 tours. Un simulateur IoT en Python a été développé pour générer des données réalistes de consommation d'eau avec des motifs dépendants de l'heure, permettant la validation et la démonstration du système. L'implémentation démontre avec succès une solution évolutive et prête pour la production avec traitement automatisé en arrière-plan, capacités de surveillance en temps réel, et interfaces utilisateur intuitives. L'évaluation des performances montre une gestion efficace de multiples capteurs concurrents avec des temps de réponse inférieurs à la seconde pour les requêtes API et une exécution fiable des tâches d'agrégation planifiées. Ce travail contribue aux infrastructures de ville intelligente en fournissant une plateforme open-source et modulaire pour la gestion des ressources en eau, avec un potentiel d'améliorations futures incluant l'analyse prédictive et la détection d'anomalies basées sur l'apprentissage automatique.

\vspace{1cm}

\medskip

\noindent
\textbf{Mots-clés :} Gestion Intelligente de l'Eau, Internet des Objets, Java Enterprise Edition, Gestion des Ressources Hydriques, API REST

\vfill

\noindent
\textbf{Compte de mots du rapport :} Environ 15 000 mots (incluant tableaux, figures, extraits de code et annexes)

\medskip

\noindent
\textbf{Dépôt GitHub :} \href{https://github.com/abderrahmanBouanani/JEE-Project-Smart-Water-Monitoring}{github.com/abderrahmanBouanani/JEE-Project-Smart-Water-Monitoring}

\medskip

\noindent

    % -------------------------------------------------------------------
    % Contents, list of figures, list of tables
    % -------------------------------------------------------------------
    \chapter*{ Liste des abréviations}
\addcontentsline{toc}{chapter}{Liste des abréviations}
%%%%%%%%%%%%%%%%%%%%%%%%%%%%%%%%%%%
%%  Enter your list of Abbreviation and Symbols in this file
%%%%%%%%%%%%%%%%%%%%%%%%%%%%%%%%%%%
Note : Trier en ordre alphabétique \\
    
\begin{abbrv}

    \item[API] Application Programming Interface (Interface de Programmation d'Application)
    
    \item[BCrypt] Blowfish Crypt (Algorithme de hachage cryptographique)
    
    \item[BPMN] Business Process Model and Notation (Modèle et Notation des Processus Métier)
    
    \item[DAO] Data Access Object (Objet d'Accès aux Données)
    
    \item[ENSAA] École Nationale Supérieure des Sciences Appliquées Agadir
    
    \item[HTTP] HyperText Transfer Protocol (Protocole de Transfert HyperTexte)
    
    \item[HTTPS] HyperText Transfer Protocol Secure (Protocole de Transfert HyperTexte Sécurisé)
    
    \item[IoT] Internet of Things (Internet des Objets)
    
    \item[Jakarta EE] Jakarta Enterprise Edition (anciennement Java EE)
    
    \item[JEE] Java Enterprise Edition (Jakarta Enterprise Edition)
    
    \item[JPA] Java Persistence API (API de Persistance Java)
    
    \item[JSON] JavaScript Object Notation (Notation d'Objet JavaScript)
    
    \item[JSP] JavaServer Pages (Pages Serveur Java)
    
    \item[MVC] Model-View-Controller (Modèle-Vue-Contrôleur)
    
    \item[MySQL] My Structured Query Language (Mon Langage de Requête Structuré)
    
    \item[OMS] Organisation Mondiale de la Santé
    
    \item[ORM] Object-Relational Mapping (Mapping Objet-Relationnel)
    
    \item[RBAC] Role-Based Access Control (Contrôle d'Accès Basé sur les Rôles)
    
    \item[REST] Representational State Transfer (Transfert d'État Représentationnel)
    
    \item[SCRUM] Framework de gestion de projet agile
    
    \item[SGBD] Système de Gestion de Base de Données
    
    \item[SQL] Structured Query Language (Langage de Requête Structuré)
    
    \item[UIZ] Université Ibn Zohr
    
    \item[UML] Unified Modeling Language (Langage de Modélisation Unifié)
    
    \item[URL] Uniform Resource Locator (Localisateur Uniforme de Ressource)
    
    \item[WAR] Web Application Archive (Archive d'Application Web)
    
    \item[XSS] Cross-Site Scripting (Script Inter-Sites)

\end{abbrv}

 %  Enter your list of Abbreviation and Symbols in this file
    \listoffigures
    \listoftables
    \tableofcontents
    %%%%%%%%%%%%%%%%%%%%%%%%%%%%%%%%%%%%%%%%%%%%%%%%%%%%%%%%%%%%%%%%%%%%%%%%
    %%                                                                    %%  
    %%  Main chapters and sections of your project                        %%  
    %%  Everything from here on needs updates in your own words and works %%
    %%                                                                    %%
    %%%%%%%%%%%%%%%%%%%%%%%%%%%%%%%%%%%%%%%%%%%%%%%%%%%%%%%%%%%%%%%%%%%%%%%%
    \mainmatter
    % Read for preparation of document in LaTex 
    % Lamport, L. (1986), LATEX: A Document Preparation System, Addison-Wesley.
    
    \chapter*{\center{Introduction Générale}}
\addcontentsline{toc}{chapter}{Introduction Générale}

\section*{Contexte et champ d'application}

L'eau représente une ressource vitale dont la gestion durable constitue l'un des défis majeurs du XXIe siècle. Selon le rapport des Nations Unies sur le développement des ressources en eau \cite{un2023water}, près de 2 milliards de personnes vivent dans des pays souffrant de stress hydrique, et cette situation ne cesse de s'aggraver avec la croissance démographique et le changement climatique. L'Organisation Mondiale de la Santé \cite{who2022water} souligne l'importance d'une surveillance continue et d'une gestion intelligente des ressources en eau pour garantir l'accès à l'eau potable et prévenir le gaspillage.

Dans ce contexte, l'Internet des Objets (IoT) et les technologies de l'information offrent des opportunités sans précédent pour révolutionner la gestion de l'eau. Les systèmes de supervision intelligente permettent de collecter, analyser et exploiter des données en temps réel pour optimiser la consommation, détecter les anomalies et sensibiliser les utilisateurs \cite{iot2020water}. Ces solutions technologiques s'inscrivent dans une démarche de développement durable et de transformation numérique des infrastructures urbaines.

\section*{Description du problème}

La gestion traditionnelle de la consommation d'eau présente plusieurs limitations majeures. Les relevés manuels des compteurs sont coûteux, peu fréquents et sujets aux erreurs humaines. L'absence de données en temps réel empêche la détection précoce des fuites et des surconsommations, entraînant un gaspillage considérable de ressources. Les consommateurs manquent de visibilité sur leurs habitudes de consommation et ne disposent pas d'outils pour optimiser leur usage de l'eau.

De plus, les gestionnaires de réseaux d'eau font face à des défis opérationnels importants : difficulté de prévision de la demande, réactivité insuffisante face aux incidents, et manque d'outils d'analyse pour la prise de décision stratégique. Ces problématiques nécessitent le développement d'une plateforme numérique capable d'intégrer des capteurs IoT, de traiter les données en temps réel, et de fournir des tableaux de bord interactifs pour tous les acteurs du système.

\section*{Objectifs du projet}

Le projet \textbf{Smart Water Monitoring System} vise à concevoir et développer une plateforme web complète de supervision intelligente de la consommation d'eau. Les objectifs spécifiques sont les suivants :

\begin{itemize}
    \item \textbf{Collecte et traitement des données IoT} : Intégrer un réseau de capteurs virtuels simulant des dispositifs IoT réels pour collecter les données de consommation d'eau en temps réel et les transmettre au système central.
    
    \item \textbf{Gestion multi-utilisateurs} : Développer un système d'authentification et d'autorisation sécurisé permettant de gérer différents profils (citoyens, administrateurs) avec des droits d'accès adaptés.
    
    \item \textbf{Visualisation et analyse} : Créer des tableaux de bord interactifs offrant une visualisation intuitive de la consommation, des tendances historiques, et des statistiques agrégées.
    
    \item \textbf{Système d'alertes intelligent} : Implémenter un mécanisme de détection automatique des anomalies (fuites, surconsommation) et de notification en temps réel des utilisateurs et des administrateurs.
    
    \item \textbf{Optimisation et objectifs de consommation} : Permettre aux utilisateurs de définir des objectifs de consommation personnalisés et de suivre leurs progrès vers une utilisation plus responsable de l'eau.
\end{itemize}

\section*{Approche et méthodologie}

Pour réaliser ce projet, nous avons adopté une approche structurée basée sur les technologies Jakarta EE et les meilleures pratiques du développement logiciel. L'architecture du système repose sur le modèle MVC (Model-View-Controller) implémenté avec Jakarta Servlet \cite{jakarta2021spec} pour la couche de contrôle, Hibernate ORM \cite{bauer2007hibernate} pour la persistance des données, et MySQL 8 \cite{mysql2023reference} comme système de gestion de base de données.

La sécurité constitue une préoccupation centrale du projet. Nous utilisons l'algorithme BCrypt \cite{bcrypt2024security} pour le hachage sécurisé des mots de passe, garantissant ainsi la protection des données sensibles des utilisateurs. Un système de filtres d'authentification contrôle l'accès aux ressources protégées et assure la séparation des privilèges entre les différents types d'utilisateurs.

Pour la simulation des capteurs IoT, nous avons développé un simulateur en Python qui génère des données réalistes de consommation d'eau en tenant compte des patterns horaires (heures de pointe, heures creuses) et qui communique avec le backend Java via des API REST. Cette approche permet de tester l'ensemble du système dans des conditions proches de la réalité sans nécessiter de matériel IoT physique.

La gestion du projet suit la méthodologie agile Scrum \cite{schwaber2020scrum}, avec des sprints de 2 à 3 semaines permettant une livraison incrémentale des fonctionnalités et une adaptation continue aux retours d'expérience.

\section*{Résultats et interprétations attendus}

À l'issue de ce projet, nous attendons de livrer une plateforme web fonctionnelle et opérationnelle capable de gérer l'ensemble du cycle de vie des données de consommation d'eau. Les résultats escomptés incluent :

\begin{itemize}
    \item Un système robuste de collecte et de stockage de données IoT avec une latence minimale et une haute disponibilité.
    \item Des interfaces utilisateur intuitives et responsives offrant une expérience utilisateur optimale sur différents supports (ordinateurs, tablettes, smartphones).
    \item Un système d'alertes efficace permettant la détection précoce des anomalies avec un taux de faux positifs maîtrisé.
    \item Des fonctionnalités d'analyse et de reporting fournissant des insights actionnables pour optimiser la consommation d'eau.
    \item Un code source bien structuré, documenté et testable, facilitant la maintenance et l'évolution future du système.
\end{itemize}

Ce projet démontre l'applicabilité des technologies Jakarta EE dans le contexte des systèmes IoT et illustre comment une approche orientée objet et une architecture en couches peuvent contribuer à développer des applications d'entreprise scalables et maintenables.

\section*{Organisation du rapport}

Le présent rapport est structuré en plusieurs chapitres complémentaires permettant une compréhension progressive du projet :

\begin{description}
    \item[Chapitre 1 -- Contexte général du projet :] Présente le cadre du projet, la problématique abordée, les objectifs visés et la démarche de gestion adoptée.
    
    \item[Chapitre 2 -- Étude préliminaire :] Détaille les concepts théoriques fondamentaux (technologies JEE, Hibernate, IoT) et l'analyse de l'état de l'art des solutions existantes.
    
    \item[Chapitre 3 -- Analyse et spécification des besoins :] Décrit les besoins fonctionnels et non fonctionnels du système, ainsi que les cas d'utilisation identifiés.
    
    \item[Chapitre 4 -- Conception de la solution :] Expose l'architecture globale du système, les diagrammes UML (classes, séquence, déploiement) et les choix de conception.
    
    \item[Chapitre 5 -- Réalisation et implémentation :] Présente les aspects techniques de l'implémentation, les extraits de code significatifs et les défis techniques rencontrés.
    
    \item[Chapitre 6 -- Tests et validation :] Décrit la stratégie de tests mise en œuvre (tests unitaires, tests d'intégration) et les résultats obtenus.
    
    \item[Chapitre 7 -- Conclusion et perspectives :] Synthétise les réalisations du projet, évalue l'atteinte des objectifs et propose des pistes d'amélioration future.
\end{description}

Ce rapport vise à fournir une vision complète et détaillée du projet, depuis sa conception initiale jusqu'à sa réalisation concrète, en mettant en lumière les choix techniques, les défis rencontrés et les solutions apportées.

    \chapter{Contexte général du projet}

\section{Introduction}
Ce chapitre a pour vocation de situer le projet dans son ensemble et de préciser 
les différents paramètres qui en encadrent la réalisation. Il permet de comprendre 
le lien avec les développements antérieurs (présentés dans le chapitre précédent, 
le cas échéant) ainsi que la façon dont il s’inscrit dans la continuité du rapport. 
Plus particulièrement:

\begin{itemize}
    \item \textbf{Rappel du contexte :} 
    Il est essentiel de rappeler les grandes lignes du domaine ou de la problématique 
    déjà abordées auparavant. Cela permet de maintenir une cohérence globale et de souligner 
    la progression logique du rapport.
    
    \item \textbf{Objectif du chapitre :} 
    Ce chapitre vise à décrire la nature du projet, sa finalité ainsi que le champ de 
    recherches ou d’applications qu’il couvre. Les informations présentées orientent et 
    justifient les choix à venir. Elles sont également nécessaires pour comprendre les 
    motivations et la portée du travail réalisé.
\end{itemize}

\section{Présentation du projet}
\subsection{Sujet du projet}
Le \textbf{choix du sujet} résulte souvent d’une commande (entreprise, institution, 
cahier des charges) ou d’une problématique identifiée (besoin d’automatisation, 
d’amélioration de processus, etc.). Les points suivants méritent d’être clarifiés :

\begin{itemize}
    \item \textbf{Origine du projet :} 
    Le contexte (académique, professionnel, industriel) dans lequel l’idée a émergé, 
    ainsi que les motivations (ex. résolution d’un problème concret, innovation technique…).
    
    \item \textbf{Domaine d’application et utilisateurs cibles :} 
    Identifier qui seront les principaux bénéficiaires ou acteurs (étudiants, professeurs, 
    clients, opérateurs, etc.). Il est important de préciser les besoins et contraintes 
    propres à ce public (environnement technique, niveau d’expertise, etc.).
\end{itemize}

\subsection{Intérêt du projet}
Cette partie met en avant la \textbf{plus-value attendue} du projet :

\begin{itemize}
    \item \textbf{Améliorations apportées :} 
    Souligner en quoi le projet se démarque de l’existant (par exemple, automatisation 
    d’une tâche auparavant manuelle, optimisation d’une application, gain de temps…).
    
    \item \textbf{Impact et pertinence :} 
    Expliquer pourquoi il est important ou nécessaire de résoudre cette problématique. 
    Dans quel cas d’usage ou contexte (universitaire, industriel, etc.) cette solution 
    présente-t-elle un véritable atout ?
\end{itemize}

\section{Problématique et état de l’existant}
Afin de \textbf{justifier} la mise en place du projet, il est souvent crucial de procéder 
à une analyse de la littérature ou des solutions déjà disponibles :

\begin{itemize}
    \item \textbf{Solutions ou méthodes proposées :} 
    Cette étape peut consister en une revue de l’existant (logiciels, algorithmes, 
    démarches) ou en l’observation d’un processus manuel devenu obsolète (par exemple, 
    saisie manuelle fastidieuse). Les principales forces et faiblesses de ces approches 
    doivent être soulignées.
    
    \item \textbf{Défauts et limites :} 
    Souligner clairement les points qui ne sont pas (ou mal) couverts par les solutions 
    actuelles : complexité excessive, coûts trop élevés, manque d’ergonomie, etc. 
    Ces lacunes mettent en évidence la \textbf{nécessité} d’une nouvelle approche 
    ou d’une amélioration.
\end{itemize}

\section{Objectifs du projet}
Les objectifs offrent un cadre structurant pour l’ensemble du travail. Ils peuvent être 
\textbf{classés en différentes catégories} :

\begin{itemize}
    \item \textbf{Objectifs fonctionnels :} 
    Il s’agit de décrire les fonctionnalités attendues du système ou de l’application. 
    Par exemple : gérer des comptes utilisateurs, effectuer un calcul d’angles, permettre 
    la visualisation de données en temps réel, etc.
    
    \item \textbf{Objectifs techniques :} 
    On aborde ici les aspects liés à la performance (temps de réponse, robustesse du système), 
    à la maintenabilité (structure modulaire, documentation), ou encore aux technologies 
    privilégiées (langages, frameworks, bases de données…). Ces objectifs peuvent inclure 
    des considérations de sécurité ou de conformité à des standards.
    
    \item \textbf{Contraintes :} 
    Les contraintes peuvent être multiples : 
    \begin{itemize}
        \item \textbf{Temporelles :} date butoir pour la livraison, temps limité pour 
        la phase de développement ou de test.
        \item \textbf{Budgétaires :} moyens financiers alloués ou ressources matérielles 
        disponibles.
        \item \textbf{Organisationnelles :} disponibilité des intervenants, politiques 
        internes de l’institution, etc.
    \end{itemize}
\end{itemize}

\section{Démarche de gestion de projet}
\subsection{Méthodologie (Scrum, Cycle en V, etc.)}
Cette partie explicite la \textbf{méthode de gestion de projet} choisie et la 
façon dont elle s’adapte au contexte :

\begin{itemize}
    \item \textbf{Présenter la méthode :} 
    Par exemple, Scrum (méthode agile) favorise l’adaptabilité et la communication 
    fréquente avec le client, tandis que le Cycle en V implique des étapes plus linéaires 
    (spécification, conception, validation).
    
    \item \textbf{Rôle des intervenants :} 
    Définir qui est le client (ou commanditaire), le product owner, l’équipe de développement, 
    etc., et expliquer brièvement les interactions (séances de feedback, revues intermédiaires…).
\end{itemize}


    La figure \ref{fig:scrum} présnete le schéma du cycle Scrum et de ses principales étapes.
    \begin{figure}[ht]
        \centering
        \includegraphics[width=0.9\linewidth]{figures/scrum.png}
        \caption{Schéma du cycle Scrum et de ses principales étapes.}
        \label{fig:scrum}
    \end{figure}

\subsection{Planification du projet}
La planification vient concrétiser la méthode sélectionnée :

\begin{itemize}
    \item \textbf{Phases ou sprints :} 
    Décrire chaque phase de manière synthétique : objectifs, durée, livrables à fournir. 
    Si vous utilisez Scrum, mentionnez le nombre de sprints et leur contenu. 
    Dans le cas d’un Cycle en V, mettez l’accent sur les étapes (spécification, conception, 
    implémentation, tests).
    
    \item \textbf{Outils de planification :} 
    Un diagramme de Gantt peut illustrer le calendrier global, tandis qu’un backlog Scrum 
    répertorie l’ensemble des tâches à accomplir. L’idéal est de décrire comment vous assurez 
    le suivi (réunions quotidiennes, rétrospectives, logiciel de gestion de tâches, etc.).
\end{itemize}

Le tableau \ref{tab:sprints} présente le planning détaillé des sprints du projet, incluant les objectifs spécifiques, la durée de chaque sprint ainsi que les livrables attendus à l'issue de chaque phase.

\begin{table}[ht]
\centering
\caption{Planification des sprints.}
\label{tab:sprints}
\begin{tabular}{|c|p{7cm}|c|p{4cm}|}
\hline
\textbf{Sprint} & \textbf{Objectifs} & \textbf{Durée} & \textbf{Livrables} \\ \hline
1 & Mise en place de l'environnement de développement, configuration des outils et préparation du dépôt de code. & 2 semaines & Environnement configuré, dépôt initial \\ \hline
2 & Développement du module d'authentification et gestion des utilisateurs. & 3 semaines & Module d'authentification opérationnel, tests unitaires \\ \hline
3 & Conception de l'interface utilisateur et début de la gestion des données. & 2 semaines & Maquettes validées, premiers écrans fonctionnels \\ \hline
4 & Intégration des fonctionnalités de base et réalisation de tests d'intégration. & 3 semaines & Application intégrée, rapport de tests \\ \hline
5 & Optimisation des performances et finalisation des modules, préparation de la version finale. & 2 semaines & Version finale livrée, documentation complète \\ \hline
\end{tabular}
\end{table}


\section{Conclusion}
En guise de synthèse, il est recommandé de récapituler les éléments clés présentés, 
afin de poser des bases solides pour la suite :

\begin{itemize}
    \item \textbf{Synthèse du contenu :} 
    Le chapitre a permis de dresser un portrait global du projet : on connaît désormais 
    ses motivations, les solutions existantes, les objectifs à atteindre et les contraintes 
    à respecter. Ces informations permettent de comprendre la pertinence et la finalité du travail.
    
    \item \textbf{Limites et difficultés rencontrées :} 
    Si certaines incertitudes ou obstacles ont été identifiés (par exemple, un manque de clarté 
    dans le cahier des charges, des variables non maîtrisées, etc.), il convient de les mentionner. 
    On peut esquisser des pistes de solutions ou décider de les aborder dans un chapitre ultérieur.
    
    \item \textbf{Transition vers le chapitre suivant :} 
    Les informations collectées et analysées ici constitueront la base de l’étude fonctionnelle 
    et/ou de la conception à venir. Le prochain chapitre pourra, par exemple, détailler 
    l’architecture envisagée, le modèle de données ou encore les diagrammes UML. 
    Cette transition assure la cohérence de la démarche globale et guide le lecteur dans la 
    progression du rapport.
\end{itemize}
 
    \chapter{Étude préliminaire et fonctionnelle}

\section{Introduction}
Ce chapitre a pour objectif de \textbf{cadrer les besoins} auxquels le projet doit répondre, 
qu’ils soient liés aux fonctionnalités principales, aux contraintes techniques ou encore 
à l’expérience utilisateur. Il permet également de clarifier \textbf{l’identité} des différents 
acteurs, ainsi que leur rôle dans le système. Enfin, il détaille la logique de fonctionnement 
au travers de cas d’utilisation et de processus métier éventuels.

\section{Étude des besoins}
Dans cette section, on distingue généralement deux grandes catégories de besoins : 
\textbf{fonctionnels} (qui décrivent ce que le système doit faire) et 
\textbf{non fonctionnels} (liés aux performances, à la fiabilité, à la sécurité, etc.).

\subsection{Besoins fonctionnels}
\begin{itemize}
    \item \textbf{Lister les fonctionnalités principales :}  
    Il s’agit de décrire toutes les actions ou services que l’application devra proposer 
    (ex. création de comptes, calcul d’un paramètre spécifique, édition de rapports, etc.).
    \item \textbf{Illustrer par des scénarios d’utilisation concrets :}  
    Décrire, par exemple, les étapes qu’un utilisateur va suivre pour atteindre un objectif 
    (inscription, consultation, validation, etc.). Cela aide à mieux cerner la séquence 
    d’actions nécessaires et l’interface ou les interfaces prévues.
\end{itemize}

\subsection{Besoins non fonctionnels}
\begin{itemize}
    \item \textbf{Contraintes de performance, de sécurité et de fiabilité :}  
    Par exemple, temps de réponse maximal acceptable, taux de disponibilité, 
    modes d’authentification ou de chiffrement des données, etc.
    \item \textbf{Implications sur le choix de l’architecture ou de la technologie :}  
    Mentionner les répercussions sur les choix de framework (Spring Boot, Django, etc.), 
    de base de données (SQL ou NoSQL), ou encore sur le design (microservices, monolithique).
\end{itemize}

\section{Identification des acteurs}
L’identification des acteurs consiste à \textbf{définir les différents profils} qui interagiront 
avec le système, ainsi que leurs droits et responsabilités.

\begin{itemize}
    \item \textbf{Description des profils :}  
    Par exemple, un administrateur, un utilisateur classique, un superviseur, un étudiant, 
    un enseignant, etc.
    \item \textbf{Fonctionnalités associées :}  
    Chaque acteur se voit attribuer des permissions spécifiques (créer un compte, valider 
    une saisie, accéder à certaines données sensibles, etc.). Lister en détail ces habilitations 
    permet d’anticiper la gestion des rôles et des autorisations.
\end{itemize}

\section{Cas d’utilisation et diagrammes}
Les \textbf{use cases} (cas d’utilisation) sont très utiles pour formaliser comment chaque acteur 
utilise le système. Ils peuvent être représentés sous forme de diagramme UML, accompagné 
d’une description textuelle plus précise.

\begin{itemize}
    \item \textbf{Présentation des use cases :}  
    Un diagramme UML permet de visualiser rapidement quelles actions sont possibles pour 
    chaque acteur. Chaque use case peut ensuite être décrit textuellement (préconditions, 
    flux principal, flux alternatif, postconditions).
    \item \textbf{Couverture des besoins fonctionnels :}  
    Vérifier que l’ensemble des fonctionnalités clés identifiées dans la section précédente 
    se retrouve bien dans les cas d’utilisation. Cela contribue à la cohérence du projet.
\end{itemize}

\section{Processus métier (si nécessaire)}
Dans certains projets, il peut être pertinent de détailler le \textbf{workflow global} (suite 
d’étapes) sous forme de \textbf{diagramme BPMN} ou de \textbf{diagramme d’activités UML}. 

\begin{itemize}
    \item \textbf{Décrire le flux d’informations :}  
    Illustrer l’enchaînement des opérations (par exemple, traitement d’une demande, validation 
    par un chef d’équipe, notification à un utilisateur, etc.).
    \item \textbf{Validations et étapes-clés :}  
    S’il y a lieu de valider un document, de signer un contrat, de changer l’état d’une commande, 
    etc., il faut expliciter ces jalons importants (qui valide ? comment ?).
\end{itemize}
La figure \ref{fig:bpmn} présente un exemple d'un diagramme BPMN.
\begin{figure}[H]
    \centering
    \includegraphics[width=0.9\linewidth]{figures/bpmn_example.png}
    \caption{BPMN Diagramme Processus Métier - Exemple Prêt de Livres. }
    \label{fig:bpmn}
\end{figure}
\section{Conclusion}
Pour conclure, il est important de dresser un bilan synthétique de l’étude fonctionnelle et 
préliminaire, afin de clarifier les points suivants :

\begin{itemize}
    \item \textbf{Résumé de la vision fonctionnelle :}  
    Rappeler brièvement les grandes fonctionnalités prévues, les rôles et responsabilités 
    des acteurs, et les contraintes majeures (performance, sécurité, etc.).
    
    \item \textbf{Transition vers l’état de l’art ou travaux préexistants :}  
    Le chapitre suivant, souvent consacré à la \textit{revue de littérature} ou à l’étude 
    de l’existant, permettra de confronter ces besoins avec les solutions ou approches déjà 
    disponibles, et d’affiner la pertinence du choix technique qui sera opéré.
\end{itemize}

    \chapter{État de l'art}

\section{Introduction}
Le présent chapitre vise à positionner le projet dans son contexte scientifique et technique, 
en s’appuyant sur un examen des approches et des travaux déjà réalisés dans le domaine. 
Il a pour rôle de \textbf{situer votre démarche} par rapport aux solutions existantes 
et d’aider à \textbf{justifier la pertinence} de votre proposition. 
Les éléments abordés peuvent inclure une recherche bibliographique, une comparaison d’outils 
ou de produits déjà disponibles, et une mise en exergue des besoins encore non satisfaits.

\section{Travaux connexes et solutions existantes}
Dans cette section, il convient de \textbf{recenser et décrire} les principaux acteurs ou 
contributions pertinentes pour le projet. Ces travaux connexes peuvent être :

\begin{itemize}
    \item \textbf{Outils ou méthodologies :} logiciels ou techniques permettant de résoudre 
    des problèmes similaires, de gérer des données comparables, ou de mettre en place 
    des processus d’automatisation. 
    \item \textbf{Articles ou études notables :} même sans citer de manière formelle, 
    on peut évoquer des recherches marquantes ayant posé des bases conceptuelles (théorie, 
    algorithmes, etc.).
    \item \textbf{Solutions industrielles ou open source :} produits ou frameworks déjà 
    disponibles sur le marché, avec leurs fonctionnalités et leur public cible.
\end{itemize}

En exposant ces différentes solutions, il est important de pointer leurs \textbf{points forts} 
et leurs \textbf{faiblesses}, par exemple :  
\begin{itemize}
    \item Atouts : rapidité, simplicité d’utilisation, maintenabilité, large communauté d’utilisateurs, etc.
    \item Limites : coûts de licence, lacunes en sécurité, manque d’extensibilité, dépendances 
    technologiques, etc.
\end{itemize}
Cette analyse permettra de préparer une \textbf{comparaison plus poussée} dans la section suivante.

\section{Analyse comparative}
Cette partie consiste à \textbf{comparer} les solutions ou approches précédemment citées 
en les évaluant selon divers critères, par exemple :

\begin{itemize}
    \item \textbf{Aspects techniques :} langage utilisé, architecture, compatibilité avec 
    d’autres systèmes, performance, évolutivité.
    \item \textbf{Aspects économiques :} coûts de mise en place, de maintenance ou de licence.
    \item \textbf{Aspects fonctionnels :} ergonomie, richesse des fonctionnalités, 
    capacité d’adaptation à différents scénarios.
\end{itemize}

Il peut être utile de synthétiser ces éléments dans un \textbf{tableau récapitulatif}, 
mettant en regard chaque solution et les critères retenus. Cette démarche met en évidence :  
\begin{itemize}
    \item \textbf{Les atouts existants} à exploiter ou à intégrer dans votre propre solution.
    \item \textbf{Les lacunes ou opportunités} de conception que vous envisagez de combler 
    (ex. proposer une interface plus intuitive, une meilleure performance, etc.).
\end{itemize}

La conclusion de cette analyse comparative consiste à \textbf{dégager la spécificité} 
ou l’originalité de votre projet. Vous pouvez y souligner si votre proposition apporte 
une innovation (méthode inédite), une performance supérieure, ou répond à un cas d’utilisation 
jusqu’ici négligé.

\section{Conclusion}
Pour conclure ce chapitre, il est essentiel de faire une \textbf{mise en perspective} :

\begin{itemize}
    \item \textbf{Bilan de l’analyse :} rappeler brièvement les solutions déjà rencontrées 
    et les raisons pour lesquelles elles peuvent être jugées incomplètes ou perfectibles 
    dans le cadre du projet.
    
    \item \textbf{Orientation :} expliquer en quoi cette étude de l’existant oriente 
    ou confirme vos choix stratégiques (technologiques, méthodologiques, etc.).
    
    \item \textbf{Transition vers le chapitre suivant (Conception) :} annoncer que 
    les conclusions tirées ici serviront de base pour \textit{élaborer une architecture 
    adaptée}, définir une modélisation adéquate (UML, diagrammes de classes, etc.), 
    ou sélectionner les outils de développement appropriés (framework web, base de données, etc.).
\end{itemize}

\noindent
Ainsi, le chapitre «État de l’art» fournit un \textbf{cadrage clair} des approches 
préexistantes et met en relief la \textbf{valeur ajoutée} que le futur système compte proposer. 
Il pave la voie au prochain chapitre, qui se concentrera sur la \textbf{conception} 
et l’architecture de la solution.

    \chapter{Analyse et conception}
\label{ch:analysis_design}

\section{Introduction}

Ce chapitre traduit les besoins fonctionnels et non fonctionnels identifiés aux chapitres précédents en une architecture technique détaillée. Il présente les choix de conception du système \textbf{Smart Water Monitoring System}, justifiés par rapport aux exigences et contraintes du projet. Nous détaillons l'approche architecturale, les diagrammes UML de modélisation, la structure des données, et les patterns de conception retenus.

\section{Approche architecturale}

\subsection{Choix d'une architecture monolithique en couches}

Le système Smart Water Monitoring adopte une \textbf{architecture monolithique en trois couches} basée sur le patron Modèle-Vue-Contrôleur (MVC) \cite{fowler2002patterns}, implémenté avec Jakarta EE (anciennement Java EE).



\begin{table}[H]
\centering
\caption{Comparaison : Architecture monolithique vs microservices pour Smart Water Monitoring.}
\label{tab:arch_comparison}
\footnotesize
\begin{tabular}{|l|l|l|}
\hline
\textbf{Critère} & \textbf{Monolithique} & \textbf{Microservices} \\ \hline
Complexité déploiement & \textbf{+} Basse (WAR unique) & Haute (multiples services) \\ \hline
Temps développement & \textbf{+} Court (cadre unifié) & Long (coordination) \\ \hline
Performance locale & \textbf{+} Optimale (pas RPC) & Dégradée (réseau) \\ \hline
Scalabilité modulaire & Limitée & \textbf{+} Excellente \\ \hline
Volume données & \textbf{+} Bon (peu de calls) & Problématique \\ \hline
\end{tabular}
\end{table}

Pour un projet académique de module JEE avec des contraintes de temps et volume de données limité, la monolithique s'avère plus appropriée \cite{richards2020architecture, newman2015microservices}. Elle permet de :
\begin{itemize}
    \item Développer rapidement avec un framework unifié (Jakarta EE).
    \item Éviter la complexité des appels réseau inter-services.
    \item Faciliter le déploiement sur un serveur unique (TomEE, WildFly, etc.).
    \item Optimiser l'accès à la base de données.
\end{itemize}

\subsection{Organisation en trois couches}

Le système est divisé en trois couches distinctes :

\subsubsection{Couche de présentation}

\textbf{Responsabilités} :
\begin{itemize}
    \item Gérer l'interface utilisateur web (formulaires, tableaux de bord, graphiques).
    \item Traiter les requêtes HTTP (GET, POST).
    \item Afficher les données et messages de rétroaction utilisateur.
\end{itemize}

\textbf{Composants} :
\begin{itemize}
    \item \textbf{Pages JSP} : \texttt{login.jsp}, \texttt{signup.jsp}, \texttt{dashboard}, etc.
    \item \textbf{Servlets} : Contrôleurs HTTP traitant les requêtes utilisateur.
    \item \textbf{Assets statiques} : CSS, JavaScript, images.
\end{itemize}

\subsubsection{Couche métier}

\textbf{Responsabilités} :
\begin{itemize}
    \item Implémenter la logique fonctionnelle du système.
    \item Gérer les règles de validation et les calculs.
    \item Orchestrer les opérations entre les DAOs et les services.
\end{itemize}

\textbf{Composants} :
\begin{itemize}
    \item \textbf{Services} : Logique métier (ex. \texttt{UtilisateurService}, \texttt{AlerteService}, \texttt{DataAggregationService}).
    \item \textbf{Modèles métier} : Classes représentant les concepts clés.
    \item \textbf{Utilitaires} : Classes d'assistance (ex. \texttt{SecurityUtil} pour BCrypt).
    \item \textbf{Jobs/Schedulers} : Tâches planifiées (ex. \texttt{DailyAggregationJob}).
\end{itemize}

\subsubsection{Couche de données}

\textbf{Responsabilités} :
\begin{itemize}
    \item Assurer la persistance et la récupération des données.
    \item Abstraire les détails d'implémentation de la base de données.
    \item Fournir une interface uniforme d'accès aux données.
\end{itemize}

\textbf{Composants} :
\begin{itemize}
    \item \textbf{DAOs (Data Access Objects)} : Accès aux données par entité.
    \item \textbf{ORM Hibernate} : Mapping objet-relationnel.
    \item \textbf{Base de données MySQL 8} : Persistance des données.
\end{itemize}

\subsection{Schéma de l'architecture}

La figure \ref{fig:architecture_layers} illustre l'organisation en couches et les flux de données.

\begin{figure}[H]
    \centering
    \includegraphics[width=0.95\linewidth]{figures/architecture_layers.png}
    \caption{Architecture en trois couches du système Smart Water Monitoring : présentation (JSP/Servlets), métier (Services), et données (DAOs/Hibernate).}
    \label{fig:architecture_layers}
\end{figure}

\section{Modèle de données}

\subsection{Entités principales}

Le système manipule 8 entités principales (+ 2 types énumérés) :

\begin{table}[H]
\centering
\caption{Entités principales du système Smart Water Monitoring.}
\label{tab:entities}
\begin{tabular}{|l|l|l|p{2.5cm}|}
\hline
\textbf{Entité} & \textbf{Clé primaire} & \textbf{Attributs clés} & \textbf{Relations} \\ \hline
Utilisateur & idUtilisateur & email, nom, motDePasse, type & 1→N Capteur, Alerte \\ \hline
CapteurIoT & idCapteur & reference, type, etat & M→1 Utilisateur \\ \hline
DonneeCapteur & idDonnee & valeur, timestamp & M→1 Capteur \\ \hline
Alerte & idAlerte & type, message, estLue & M→1 Utilisateur \\ \hline
ObjectifConsommation & idObjectif & valeurObjectif, mois & M→1 Utilisateur \\ \hline
HistoriqueConsommation & idHistorique & consommationTotal, date & M→1 Utilisateur \\ \hline
Statistique & idStatistique & consommationMoyenne & M→1 Utilisateur \\ \hline
TypeAlerte & idType & nomType, description & 1→N Alerte \\ \hline
\end{tabular}
\end{table}

\subsection{Diagramme de classes}

La figure \ref{fig:class_diagram} présente le diagramme UML des classes du domaine.

\begin{figure}[H]
    \centering
    \includegraphics[width=0.95\linewidth]{figures/WMS_class_diagram.png}
    \caption{Diagramme UML des classes du système montrant les entités, leurs attributs, et les relations de multiplicité.}
    \label{fig:class_diagram}
\end{figure}

\subsection{Schéma de la base de données}

Le tableau \ref{tab:db_schema} détaille la structure des tables principales :

\begin{table}[H]
\centering
\caption{Structure de la base de données MySQL 8 - Tables principales.}
\label{tab:db_schema}
\begin{tabular}{|l|l|l|}
\hline
\textbf{Table} & \textbf{Colonne} & \textbf{Type / Contrainte} \\ \hline
\multirow{6}{*}{\textbf{utilisateurs}} & idUtilisateur & INT PRIMARY KEY AUTO\_INCREMENT \\ \cline{2-3}
& email & VARCHAR(100) UNIQUE NOT NULL \\ \cline{2-3}
& motDePasse & VARCHAR(255) NOT NULL (BCrypt) \\ \cline{2-3}
& type & ENUM('CITOYEN', 'ADMINISTRATEUR') \\ \cline{2-3}
& dateInscription & TIMESTAMP DEFAULT CURRENT\_TIMESTAMP \\ \hline
\multirow{5}{*}{\textbf{capteurs}} & idCapteur & INT PRIMARY KEY AUTO\_INCREMENT \\ \cline{2-3}
& reference & VARCHAR(100) UNIQUE NOT NULL \\ \cline{2-3}
& type & ENUM('EAU\_FROIDE', 'EAU\_CHAUDE', 'TOTAL') \\ \cline{2-3}
& utilisateur\_id & INT FOREIGN KEY → utilisateurs \\ \cline{2-3}
& etat & BOOLEAN DEFAULT TRUE \\ \hline
\multirow{4}{*}{\textbf{donnees\_capteurs}} & idDonnee & INT PRIMARY KEY AUTO\_INCREMENT \\ \cline{2-3}
& capteur\_id & INT FOREIGN KEY → capteurs \\ \cline{2-3}
& valeur & DECIMAL(8,2) NOT NULL \\ \cline{2-3}
& timestamp & TIMESTAMP DEFAULT CURRENT\_TIMESTAMP \\ \hline
\multirow{3}{*}{\textbf{alertes}} & idAlerte & INT PRIMARY KEY AUTO\_INCREMENT \\ \cline{2-3}
& utilisateur\_id & INT FOREIGN KEY → utilisateurs \\ \cline{2-3}
& type & ENUM('FUITE', 'SURCONSOMMATION', 'ANOMALIE') \\ \hline
\end{tabular}
\end{table}

\section{Diagrammes de séquence}

\subsection{Séquence d'authentification}

La figure \ref{fig:sequence_auth} illustre le flux d'authentification.

\begin{figure}[H]
    \centering
    \includegraphics[width=0.9\linewidth]{figures/sequence_auth.png}
    \caption{Diagramme de séquence : Authentification utilisateur avec vérification BCrypt.}
    \label{fig:sequence_auth}
\end{figure}

\subsection{Séquence de collecte IoT}

La figure \ref{fig:sequence_iot} illustre le flux de collecte de données.

\begin{figure}[H]
    \centering
    \includegraphics[width=0.9\linewidth]{figures/sequence_iot.png}
    \caption{Diagramme de séquence : Collecte des données IoT via API REST.}
    \label{fig:sequence_iot}
\end{figure}

\subsection{Séquence d'agrégation quotidienne}

La figure \ref{fig:sequence_aggregation} illustre le processus d'agrégation des données quotidiennes.

\begin{figure}[H]
    \centering
    \includegraphics[width=0.9\linewidth]{figures/sequence_aggregation.png}
    \caption{Diagramme de séquence : Job d'agrégation quotidienne des données de consommation.}
    \label{fig:sequence_aggregation}
\end{figure}

\subsection{Workflow général du système}

La figure \ref{fig:wms_sequence} illustre le workflow complet depuis l'authentification jusqu'à la notification des alertes.

\begin{figure}[H]
    \centering
    \includegraphics[width=0.95\linewidth]{figures/WMS_sequence_diagram.png}
    \caption{Diagramme de séquence : Workflow général du système incluant authentification, consulta des données, vérification des objectifs, et génération d'alertes.}
    \label{fig:wms_sequence}
\end{figure}

\section{Patterns de conception}

Le système utilise plusieurs patterns reconnus \cite{fowler2002patterns} :

\subsection{Pattern DAO}
Abstrait l'accès aux données via \texttt{AbstractDao<T>} et DAOs spécialisés.

\subsection{Pattern Service}
Encapsule la logique métier via \texttt{IService<T>} et services implémentants.

\subsection{Pattern Singleton}
Garantit une instance unique pour \texttt{DailyAggregationJob} et \texttt{HibernateUtil}.

\subsection{Pattern MVC}
Sépare Model (JPA), View (JSP), et Controller (Servlets).

\subsection{Pattern Filtre}
\texttt{AuthenticationFilter} intercepte et contrôle les requêtes HTTP.

\section{Sécurité}

\subsection{Hachage des mots de passe}
Utilise BCrypt avec 12 rounds de salage.

\subsection{Gestion de session}
Sessions HTTP côté serveur avec timeout configurable (30 minutes).

\subsection{Contrôle d'accès (RBAC)}
Deux rôles (CITOYEN, ADMINISTRATEUR) avec filtres d'accès.

\section{Pile technologique}

Le tableau \ref{tab:tech_stack} résume les technologies utilisées.

\begin{table}[H]
\centering
\caption{Pile technologique du système Smart Water Monitoring.}
\label{tab:tech_stack}
\begin{tabular}{|l|l|l|}
\hline
\textbf{Couche} & \textbf{Technologie} & \textbf{Version} \\ \hline
Présentation & Jakarta Servlet / JSP & 5.0.0 / 2.0.0 \\ \hline
Métier & Jakarta EE & 9.1 \\ \hline
ORM & Hibernate & 6.4.4 \\ \hline
SGBD & MySQL & 8.0 \\ \hline
Sécurité & jBCrypt & 0.4 \\ \hline
Build & Maven & 3.x \\ \hline
\end{tabular}
\end{table}

\section{Conclusion}

Ce chapitre a détaillé la conception du système Smart Water Monitoring avec architecture monolithique en trois couches, modèle de données robuste, patterns SOLID, et sécurité renforcée. Ces choix assurent une base solide pour l'implémentation qui suit.

Le formalisme UML utilisé (diagrammes de classes, de séquence, cas d'utilisation) a permis de modéliser clairement la structure et le comportement du système, facilitant la communication au sein de l'équipe et garantissant la cohérence du projet. L'architecture en couches adoptée répond aux besoins fonctionnels et non fonctionnels identifiés, en assurant maintenabilité, modularité et sécurité. Les modèles présentés dans ce chapitre serviront de base pour l'implémentation effective du système, qui sera détaillée dans le chapitre suivant.

    \chapter{Technologies et réalisation}

Ce chapitre décrit l'ensemble des technologies, outils et méthodes qui ont été utilisés pour la mise en œuvre du projet. Il présente d'abord les choix techniques, puis détaille l'environnement de déploiement, l'implémentation du système (structure du code et interfaces utilisateur), ainsi que la stratégie de tests appliquée. Enfin, une conclusion synthétique fait le point sur les réalisations et les éventuelles difficultés rencontrées.

%%%%%%%%%%%%%%%%%%%%%%%%%%%%%%%%%%%%%%%%%%%%%%%%%%%%%%%%%%%
\section{Introduction}
%%%%%%%%%%%%%%%%%%%%%%%%%%%%%%%%%%%%%%%%%%%%%%%%%%%%%%%%%%%
Dans cette section introductive, l'objectif est de donner une vue d'ensemble des aspects techniques et pratiques abordés dans ce chapitre. On y explique notamment :
\begin{itemize}
    \item Les choix technologiques réalisés en fonction des besoins fonctionnels et des contraintes techniques.
    \item La manière dont l'environnement de déploiement est configuré pour supporter le système.
    \item Les détails de l'implémentation du code et de l'interface utilisateur.
    \item La stratégie adoptée pour valider et tester le système.
\end{itemize}
Cette partie prépare le lecteur à comprendre comment les décisions prises au niveau de la conception se traduisent concrètement lors de la réalisation.

%%%%%%%%%%%%%%%%%%%%%%%%%%%%%%%%%%%%%%%%%%%%%%%%%%%%%%%%%%%
\section{Choix techniques et outils}
%%%%%%%%%%%%%%%%%%%%%%%%%%%%%%%%%%%%%%%%%%%%%%%%%%%%%%%%%%%
Cette section détaille les principales technologies et outils qui ont été sélectionnés pour le développement du projet.

\subsection{Langages, Frameworks, Bases de données…}
\begin{itemize}
    \item \textbf{Langages de programmation :}  
    Justifier le choix du ou des langages (par exemple, Java pour la robustesse et la scalabilité, Python pour la rapidité de développement ou JavaScript/React pour le front-end interactif).
    \item \textbf{Frameworks :}  
    Expliquer la sélection des frameworks, tels que Spring Boot pour le back-end, React ou Angular pour le front-end, ou encore d'autres frameworks spécifiques à la tâche.
    \item \textbf{Bases de données :}  
    Préciser le type de base de données utilisée (relationnelle comme MySQL/PostgreSQL ou NoSQL comme MongoDB) en fonction des critères de performance, de flexibilité et de volume de données.
    \item \textbf{Critères de sélection :}  
    Indiquer les critères qui ont guidé ces choix, par exemple la taille de la communauté, la robustesse, la simplicité d'intégration et la performance.
\end{itemize}
Exemple d'utilisation des références : 
Dans ce rapport, diverses références ont été utilisées pour étayer la démarche adoptée. Par exemple, la conception de sites web modernes est détaillée dans \cite{duckett2011html}, tandis que \cite{horstmann2014jee} offre une vue d'ensemble de l'architecture Java EE. Pour une approche approfondie de l'ORM avec Hibernate, on peut se référer à \cite{bauer2007hibernate}. Enfin, \cite{keogh2002jsp} fournit un guide complet sur le développement de JavaServer Pages.

\subsection{Outils de développement (IDE, gestion de versions…)}
\begin{itemize}
    \item \textbf{Environnement de développement intégré (IDE) :}  
    Mentionner les outils utilisés, par exemple VS Code, Eclipse ou IntelliJ IDEA, et expliquer pourquoi ces outils ont été privilégiés.
    \item \textbf{Système de gestion de versions :}  
    Indiquer l'utilisation d'un outil comme Git (hébergé sur GitHub ou GitLab) pour assurer le suivi des modifications, faciliter la collaboration et la gestion du code source.
    \item \textbf{Autres outils de collaboration et de tests :}  
    Par exemple, l'utilisation de Maven ou Gradle pour la gestion de projet, ainsi que des outils de CI/CD pour automatiser les builds et les tests.
\end{itemize}

%%%%%%%%%%%%%%%%%%%%%%%%%%%%%%%%%%%%%%%%%%%%%%%%%%%%%%%%%%%
\section{Architecture de déploiement}
%%%%%%%%%%%%%%%%%%%%%%%%%%%%%%%%%%%%%%%%%%%%%%%%%%%%%%%%%%%
Cette section décrit l'infrastructure sur laquelle le système sera déployé et les outils utilisés pour assurer son fonctionnement continu.

\begin{itemize}
    \item \textbf{Environnement cible :}  
    Décrire le ou les serveurs, l'utilisation éventuelle de conteneurs (Docker) ou du cloud (AWS, Azure, Google Cloud), ainsi que les systèmes d'exploitation concernés.
    \item \textbf{Pipeline d'intégration continue (CI/CD) :}  
    Indiquer si un pipeline CI/CD a été mis en place pour automatiser les phases de build, de tests et de déploiement, et présenter les outils utilisés (Jenkins, GitLab CI, Travis CI, etc.).
\end{itemize}

%%%%%%%%%%%%%%%%%%%%%%%%%%%%%%%%%%%%%%%%%%%%%%%%%%%%%%%%%%%
\section{Implémentation et interfaces}
%%%%%%%%%%%%%%%%%%%%%%%%%%%%%%%%%%%%%%%%%%%%%%%%%%%%%%%%%%%
Cette section est consacrée à la description de l'implémentation du système ainsi qu'à la présentation des interfaces utilisateur.

\begin{itemize}
    \item \textbf{Structure du code :}  
    Exposer l'organisation du code en packages ou modules. Décrire brièvement la répartition entre le front-end, la logique métier et la couche de données.
    \item \textbf{Interfaces utilisateur :}  
    Présenter l’interface graphique à travers des captures d’écran ou des maquettes. Décrire les principales fonctionnalités disponibles pour l’utilisateur final.
    \item \textbf{Fonctionnalités principales :}  
    Détaillez les fonctionnalités clés implémentées, tant côté front-end (navigation, formulaires, affichage des données) que côté back-end (gestion des requêtes, traitement des données, sécurité).
\end{itemize}

%%%%%%%%%%%%%%%%%%%%%%%%%%%%%%%%%%%%%%%%%%%%%%%%%%%%%%%%%%%
\section{Stratégie de tests}
%%%%%%%%%%%%%%%%%%%%%%%%%%%%%%%%%%%%%%%%%%%%%%%%%%%%%%%%%%%
Pour garantir la qualité et la robustesse du système, une stratégie de tests complète a été mise en place.

\begin{itemize}
    \item \textbf{Tests unitaires :}  
    Décrire la couverture des tests unitaires qui vérifient le fonctionnement de chaque composant individuel (par exemple, avec JUnit, pytest, etc.).
    \item \textbf{Tests d'intégration :}  
    Expliquer comment les différents modules du système ont été testés ensemble pour s'assurer de la bonne communication entre eux.
    \item \textbf{Tests système :}  
    Préciser si des tests globaux ont été effectués pour valider le comportement complet du système, incluant l’interface utilisateur.
    \item \textbf{Outils de test et plan de validation :}  
    Mentionner les outils utilisés (Selenium pour les tests d’interface, Postman pour les API, etc.) et résumer le plan de validation ainsi que les résultats obtenus de manière synthétique.
\end{itemize}

%%%%%%%%%%%%%%%%%%%%%%%%%%%%%%%%%%%%%%%%%%%%%%%%%%%%%%%%%%%
\section{Conclusion}
%%%%%%%%%%%%%%%%%%%%%%%%%%%%%%%%%%%%%%%%%%%%%%%%%%%%%%%%%%%
Pour clore ce chapitre, il est important de faire un bilan sur l’ensemble des décisions techniques et 
de réalisation qui ont été prises :

\begin{itemize}
    \item \textbf{Synthèse des choix techniques :}  
    Récapituler les technologies, outils et méthodologies adoptés, en insistant sur leur adéquation 
    avec les besoins du projet (simplicité d’utilisation, performance, évolutivité, etc.).
    \item \textbf{Retour sur l'architecture de déploiement et l'implémentation :}  
    Expliquer brièvement comment l'environnement de déploiement et la structure du code assurent la 
    stabilité et la pérennité du système.
    \item \textbf{Perspectives et défis :}  
    Mentionner les difficultés éventuelles rencontrées lors de l’implémentation (limitations techniques, 
    problèmes d’intégration, etc.) et indiquer les pistes d’amélioration à envisager pour les phases ultérieures.
    \item \textbf{Transition vers le chapitre suivant :}  
    Annoncer que les éléments techniques présentés ici constitueront la base pour l’implémentation 
    concrète du système, qui sera détaillée dans le chapitre de réalisation pratique.
\end{itemize}

\noindent
Ce chapitre constitue ainsi un socle technique solide, garantissant que les choix de développement 
sont cohérents avec les objectifs du projet et prêts à être mis en œuvre lors de la phase de réalisation.

    \chapter*{ \center Conclusion générale}
\addcontentsline{toc}{chapter}{Conclusion générale}

La conclusion générale met en lumière les principaux enseignements et aboutissements du projet. Elle récapitule les objectifs initiaux, synthétise les contributions et résultats obtenus, identifie les limites du travail réalisé, et propose des pistes pour des développements futurs. 

\begin{itemize}
    \item \textbf{Récapitulation des objectifs initiaux :}
    \begin{itemize}
        \item Les problématiques et questions de recherche abordées dès le départ sont rappelées afin de souligner l'enjeu initial du projet.
        \item Une évaluation est présentée pour vérifier dans quelle mesure les objectifs techniques et fonctionnels ont été atteints.
    \end{itemize}
    
    \item \textbf{Contributions et résultats obtenus :}
    \begin{itemize}
        \item Les réalisations concrètes du projet (fonctionnalités développées, prototypes validés, analyses réalisées, etc.) sont synthétisées.
        \item Les apports spécifiques du projet sont mis en avant, qu'il s'agisse d'innovations techniques, de gains de temps ou d'améliorations de performance.
    \end{itemize}
    
    \item \textbf{Limites du travail :}
    \begin{itemize}
        \item Les points de blocage et contraintes rencontrés (budgétaires, délais, aspects techniques) sont identifiés.
        \item Une analyse critique montre en quoi ces limites peuvent influencer l'utilisation ou l'évolution du système développé.
    \end{itemize}
    
    \item \textbf{Perspectives et améliorations futures :}
    \begin{itemize}
        \item Des pistes d'évolution sont proposées, telles que l'ajout de nouvelles fonctionnalités, l'optimisation des performances, ou la migration vers des technologies plus avancées.
        \item Pour les projets à dimension scientifique, des axes de recherche complémentaires sont suggérés pour approfondir les résultats obtenus.
    \end{itemize}
\end{itemize}


    \chapter*{ \center Apport du projet}
\label{ch:apport}

Ce projet de Smart Water Monitoring a été une expérience formatrice sur les plans technique et personnel. Il nous a permis de développer une approche méthodique pour résoudre des problèmes complexes dans un contexte réel.

Sur le plan technique, nous avons acquis une maîtrise concrète des technologies Java EE, Hibernate et MySQL pour le développement d'applications web. L'intégration d'un simulateur IoT en Python a renforcé nos compétences en interopérabilité entre différents langages. La gestion de la sécurité avec BCrypt et le contrôle d'accès nous a sensibilisés aux enjeux de protection des données.

Sur le plan méthodologique, ce projet nous a appris à organiser notre travail en suivant une approche structurée, depuis l'analyse des besoins jusqu'aux tests finaux. La rédaction du rapport en LaTeX a développé notre rigueur dans la documentation technique. Nous avons également appris à nous adapter face aux défis techniques et à prioriser les fonctionnalités essentielles.

Les principaux défis ont concerné l'optimisation des performances de l'agrégation des données et la gestion des communications entre les différents composants du système. Ces difficultés nous ont appris à rechercher des solutions alternatives et à persévérer face aux obstacles techniques.

Si c'était à refaire, nous accorderions plus de temps à la phase de planification initiale et nous documenterions mieux les décisions techniques au fur et à mesure du développement. Nous mettrions également en place plus de tests automatisés pour faciliter les modifications futures.

Cette expérience a renforcé notre confiance dans notre capacité à mener un projet complexe du début à la fin, et a consolidé notre intérêt pour le développement d'applications utiles à l'environnement et à la société.
    
    % -------------------------------------------------------------------
    % Bibliography/References  -  Harvard Style was used in this report
    % -------------------------------------------------------------------
    %\bibliography{references}  %  Patashnik, O. (1988), BibTEXing. Documentation for general BibTEX users.
    
    % -------------------------------------------------------------------
    % Appendices
    % -------------------------------------------------------------------
    
    \begin{appendices}
        \chapter{Annexes (Facultatif)}
\label{appn:A}

Les annexes regroupent des documents complémentaires — tels que des tableaux volumineux, des extraits de code, des données brutes, des démonstrations détaillées ou d'autres preuves — qui, bien qu'importants pour une compréhension approfondie du projet, ne constituent pas l'essence même du rapport. Une annexe est destinée aux lecteurs souhaitant obtenir des informations supplémentaires ou une explication plus complète sur certains aspects du projet. Il est recommandé d'utiliser une annexe par idée ou par thème afin de maintenir une organisation claire et structurée.

L'inclusion d'annexes est facultative. Si vous estimez que les informations supplémentaires ne sont pas indispensables à la compréhension du projet, il est préférable de ne pas les ajouter. En effet, inclure des documents non pertinents ou superflus pourrait augmenter indûment le nombre total de pages du rapport et distraire le lecteur.

        \chapter{Annexes (Facultatif)}
\label{appn:B}


...
    \end{appendices}
  
\newpage
\thispagestyle{empty}
\begin{center}
  \begin{tcolorbox}[arc=5mm, colframe=black, colback=white, left=5pt, right=5pt, top=20pt, bottom=20pt, width=0.9\textwidth]
    \begin{center}
      
      {\Large \textbf{Abderrahman Bouanani}}\\[0.5cm]
      {\Large \textbf{Résumé du Projet}}\\[1cm]
    \end{center}
    
    \begin{quote}
      Le présent projet vise à développer une solution innovante destinée à [décrire brièvement l'objectif principal du projet]. Pour atteindre cet objectif, une approche méthodologique rigoureuse a été adoptée, combinant une analyse fonctionnelle détaillée, une conception modulaire et l'implémentation de technologies modernes.
      
      Les contributions majeures du projet incluent la mise en œuvre d'une interface utilisateur intuitive, l'intégration d'un système performant de gestion des données et la validation de la solution via des tests exhaustifs. Ces résultats traduisent une amélioration significative en termes de [performance, réactivité, sécurité, etc.] par rapport aux solutions existantes.
      
      Ce projet a permis d'acquérir des compétences avancées en [développement web, modélisation UML, architecture logicielle, etc.], tout en consolidant la compréhension de la gestion de projet dans un environnement technique exigeant. En outre, il ouvre des perspectives prometteuses pour de futurs développements, tels que l'ajout de nouvelles fonctionnalités, l'optimisation des performances et l'amélioration de l'ergonomie de l'interface utilisateur.
      
      L'ensemble de ces réalisations témoigne de l'engagement à proposer une solution à la fois innovante et efficace, répondant aux problématiques identifiées dès le début du projet.
    \end{quote}
    
    \vspace{1cm}
    \noindent \textbf{Mots clés :} solution innovante, gestion des données, interface utilisateur, tests exhaustifs, optimisation des performances
  \end{tcolorbox}
\end{center}
\end{document}
