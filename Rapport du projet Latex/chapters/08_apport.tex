\chapter*{ \center Apport du projet}
\label{ch:apport}

Ce projet de Smart Water Monitoring a été une expérience formatrice sur les plans technique et personnel. Il nous a permis de développer une approche méthodique pour résoudre des problèmes complexes dans un contexte réel.

Sur le plan technique, nous avons acquis une maîtrise concrète des technologies Java EE, Hibernate et MySQL pour le développement d'applications web. L'intégration d'un simulateur IoT en Python a renforcé nos compétences en interopérabilité entre différents langages. La gestion de la sécurité avec BCrypt et le contrôle d'accès nous a sensibilisés aux enjeux de protection des données.

Sur le plan méthodologique, ce projet nous a appris à organiser notre travail en suivant une approche structurée, depuis l'analyse des besoins jusqu'aux tests finaux. La rédaction du rapport en LaTeX a développé notre rigueur dans la documentation technique. Nous avons également appris à nous adapter face aux défis techniques et à prioriser les fonctionnalités essentielles.

Les principaux défis ont concerné l'optimisation des performances de l'agrégation des données et la gestion des communications entre les différents composants du système. Ces difficultés nous ont appris à rechercher des solutions alternatives et à persévérer face aux obstacles techniques.

Si c'était à refaire, nous accorderions plus de temps à la phase de planification initiale et nous documenterions mieux les décisions techniques au fur et à mesure du développement. Nous mettrions également en place plus de tests automatisés pour faciliter les modifications futures.

Cette expérience a renforcé notre confiance dans notre capacité à mener un projet complexe du début à la fin, et a consolidé notre intérêt pour le développement d'applications utiles à l'environnement et à la société.