\chapter*{Remerciements}
\addcontentsline{toc}{chapter}{Remerciements}

Il est d’usage d’exprimer sa gratitude envers toutes les personnes et organismes ayant 
contribué à la bonne réalisation du projet. Les remerciements peuvent inclure, entre autres, 
l’encadrant ou les encadrants pour leur accompagnement, les amis et collègues pour leurs 
soutiens, ainsi que toute instance (département, laboratoire, institut, etc.) ayant mis 
des ressources ou des installations à disposition. Cette section reste facultative, mais 
elle constitue un espace pour reconnaître publiquement toute aide ou appui dont le projet 
a bénéficié.
