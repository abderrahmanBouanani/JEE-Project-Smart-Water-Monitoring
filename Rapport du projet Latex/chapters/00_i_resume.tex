donner en anglais  : \chapter*{Résumé}
\addcontentsline{toc}{chapter}{Résumé}

Ce document propose un modèle de rapport de projet ainsi que des indications sur la manière 
de rédiger un rapport. Il inclut également plusieurs exemples utiles permettant de 
s’habituer à \LaTeX. Le nombre et l’intitulé des chapitres peuvent varier selon le type 
de projet et les préférences de chacun. Les titres de section présentés ici ne sont qu’illustratifs 
et doivent être adaptés. De même, le nombre de sections dans chaque chapitre reste flexible. 
Il est possible que ce modèle convienne plus ou moins à votre projet ; il est donc conseillé 
de discuter de la structure du rapport avec votre encadrant.

\medskip

\noindent\textbf{Conseils sur la rédaction d’un résumé :}  
Le résumé est une synthèse du rapport, présentée en un seul paragraphe, d’une longueur 
maximale de 250 mots. Il doit être \textit{autonome} et ne doit pas faire référence 
à d’autres sections, figures, tableaux, équations ou références. En général, un résumé 
comprend quatre éléments clés : 
\begin{enumerate}
    \item \textbf{Introduction} (contexte et objectif du projet) ;
    \item \textbf{Méthodes} (expérimentations, techniques ou implémentation) ;
    \item \textbf{Résultats} (principales conclusions obtenues, ainsi que leur portée) ;
    \item \textbf{Conclusions} (implications pour le domaine d’étude).
\end{enumerate}
La répartition de ces quatre parties doit refléter l’importance relative de chacune 
dans le corps du rapport. Le résumé débute par quelques phrases décrivant la thématique 
générale et l’objectif principal du projet, puis présente la problématique ciblée. 
Ensuite, une courte description de la méthodologie employée est donnée, avant de rappeler 
les résultats obtenus et leur signification. Enfin, la conclusion souligne la contribution 
et les retombées majeures pour le domaine concerné.

\vspace{1cm}
\noindent
\textbf{Mots-clés :} maximum de cinq mots-clés ou expressions-clés, séparés par des virgules

\vfill

\noindent
\textbf{Compte de mots du rapport :} Le nombre de mots doit être indiqué après le résumé. 
Le rapport doit comporter au moins 10\,000 mots et au maximum 15\,000 mots (en comptant 
du premier chapitre jusqu’à la fin du chapitre de conclusion, mais en excluant la liste des références, 
les annexes, le résumé ainsi que les textes contenus dans les figures, tableaux, listings et légendes), 
soit environ 40 à 50 pages.

\medskip

\noindent
\textbf{Le code source du projet doit être déposé sur GitLab. Le lien GitLab doit figurer ici, après le décompte de mots.}

\medskip

\noindent
Le rapport (idéalement en format PDF) doit être remis via la plateforme « Ecampus »  avant la date limite. Si un(e) étudiant(e) a des examens de rattrapage 
pour les modules enseignés, la date limite de remise du mémoire sera repoussée de deux semaines 
par rapport à la date initiale.