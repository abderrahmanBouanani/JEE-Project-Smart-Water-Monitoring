\chapter*{Résumé}
\addcontentsline{toc}{chapter}{Résumé}

La rareté de l'eau et la gestion inefficace des ressources posent des défis majeurs pour le développement urbain durable. Ce projet présente Smart Water Monitoring, une plateforme web complète pour la gestion intelligente de la consommation d'eau utilisant la technologie Internet des Objets (IoT) et Java Enterprise Edition (JEE). Le système répond au besoin critique de surveillance en temps réel, d'analyse automatisée des données et de gestion proactive des alertes pour optimiser l'utilisation des ressources hydriques dans les environnements résidentiels. La plateforme implémente une architecture JEE multi-tiers suivant le patron Modèle-Vue-Contrôleur, utilisant l'API Jakarta Servlet pour le traitement des requêtes, Hibernate ORM pour la persistance des données, et MySQL 8 pour le stockage. Les fonctionnalités principales incluent la collecte de données de capteurs IoT en temps réel via des API REST, l'agrégation quotidienne automatique de la consommation par tâches planifiées, la génération intelligente d'alertes basée sur des seuils configurables et des motifs temporels, l'analyse statistique complète, et le contrôle d'accès basé sur les rôles pour administrateurs et citoyens. Les mesures de sécurité intègrent le hachage BCrypt des mots de passe avec salage à 12 tours. Un simulateur IoT en Python a été développé pour générer des données réalistes de consommation d'eau avec des motifs dépendants de l'heure, permettant la validation et la démonstration du système. L'implémentation démontre avec succès une solution évolutive et prête pour la production avec traitement automatisé en arrière-plan, capacités de surveillance en temps réel, et interfaces utilisateur intuitives. L'évaluation des performances montre une gestion efficace de multiples capteurs concurrents avec des temps de réponse inférieurs à la seconde pour les requêtes API et une exécution fiable des tâches d'agrégation planifiées. Ce travail contribue aux infrastructures de ville intelligente en fournissant une plateforme open-source et modulaire pour la gestion des ressources en eau, avec un potentiel d'améliorations futures incluant l'analyse prédictive et la détection d'anomalies basées sur l'apprentissage automatique.

\vspace{1cm}

\medskip

\noindent
\textbf{Mots-clés :} Gestion Intelligente de l'Eau, Internet des Objets, Java Enterprise Edition, Gestion des Ressources Hydriques, API REST

\vfill

\noindent
\textbf{Compte de mots du rapport :} Environ 15 000 mots (incluant tableaux, figures, extraits de code et annexes)

\medskip

\noindent
\textbf{Dépôt GitHub :} \href{https://github.com/abderrahmanBouanani/JEE-Project-Smart-Water-Monitoring}{github.com/abderrahmanBouanani/JEE-Project-Smart-Water-Monitoring}

\medskip

\noindent
