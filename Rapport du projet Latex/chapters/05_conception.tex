\chapter{Analyse et conception}
\label{ch:analysis_design}

% =============================
% INTRODUCTION
% =============================
\section{Introduction}
Cette section présente l’objectif du chapitre, qui consiste à traduire les besoins fonctionnels 
et non fonctionnels du projet en une solution technique détaillée. Elle explique comment les choix 
de modélisation, d’architecture et de conception ont été déterminés et justifiés, et comment ils 
s’inscrivent dans la logique globale du projet. 

% =============================
% FORMALISME DE MODÉLISATION
% =============================
\section{Formalisme de modélisation (UML, etc.)}
\subsection*{Motivation et choix}
Décrivez ici pourquoi vous avez choisi d’utiliser un langage de modélisation tel que l’UML.  
Précisez en quoi ce formalisme aide à :
\begin{itemize}
    \item Décrire la structure et le comportement du système.
    \item Communiquer efficacement entre les membres de l’équipe.
    \item Garantir la cohérence et la maintenabilité du projet.
\end{itemize}

\subsection*{Diagrammes retenus}
Listez et décrivez les différents diagrammes que vous avez retenus pour modéliser le système, par exemple :
\begin{itemize}
    \item \textbf{Diagramme de classes :} pour représenter la structure statique (entités, attributs, méthodes, relations).
    \item \textbf{Diagrammes de séquence :} pour illustrer l’enchaînement des interactions lors de scénarios clés.
    \item \textbf{Autres diagrammes (activités, composants, déploiement) :} selon les besoins du projet.
\end{itemize}

% =============================
% ARCHITECTURE LOGICIELLE
% =============================
\section{Architecture logicielle}
Cette section décrit l’\textbf{architecture globale} du système, en distinguant clairement 
l’approche adoptée et en expliquant la répartition fonctionnelle et technique du code.

\subsection{Approche architecturale}
Présentez ici le choix entre une architecture \textbf{monolithique} et une architecture 
\textbf{microservices} en détaillant :
\begin{itemize}
    \item \textbf{Architecture monolithique :} 
    \begin{itemize}
        \item \emph{Avantages} : simplicité de développement initial, déploiement unifié.
        \item \emph{Inconvénients} : difficulté de maintenance et de scalabilité sur le long terme.
    \end{itemize}
    \item \textbf{Architecture microservices :} 
    \begin{itemize}
        \item \emph{Avantages} : scalabilité fine, isolation des pannes, flexibilité technologique.
        \item \emph{Inconvénients} : complexité de déploiement et de gestion des communications inter-services.
    \end{itemize}
\end{itemize}
Justifiez le choix fait pour votre projet en fonction des exigences et contraintes identifiées.

\subsection{Schéma de l'architecture et organisation en couches}
Expliquez la division du système en différentes couches, par exemple :
\begin{itemize}
    \item \textbf{Couche de présentation :} gère l’interface utilisateur.
    \item \textbf{Couche de logique métier :} contient les règles de traitement et la gestion des opérations.
    \item \textbf{Couche de données :} responsable de la persistance et de l’accès aux informations.
\end{itemize}
Vous pouvez également présenter un schéma illustrant les interactions entre ces couches.

\subsection{Discussion}
Précisez en quoi cette architecture répond aux besoins fonctionnels et non fonctionnels du projet 
(maintenabilité, scalabilité, performance, sécurité, etc.) et comment elle prépare la voie pour l’implémentation.

% =============================
% DIAGRAMMES DE CONCEPTION
% =============================
\section{Diagrammes de conception}
Cette section détaille les différents diagrammes qui ont servi à formaliser la conception du système.

\subsection{Diagramme de classes}
\begin{itemize}
    \item \textbf{Présentation des entités :} listez les classes principales, leurs attributs et leurs méthodes.
    \item \textbf{Relations et cardinalités :} décrivez les associations, agrégations, compositions et héritages, en précisant les cardinalités.
\end{itemize}

\subsection{Diagrammes de séquence}
\begin{itemize}
    \item \textbf{Scénarios clés :} illustrez l’enchaînement des interactions pour des processus importants (exemple : authentification, traitement d'une requête, etc.).
    \item \textbf{Flux d’interaction :} montrez comment les messages circulent entre les objets ou composants pour répondre à une action utilisateur.
\end{itemize}

\subsection{Autres diagrammes utiles}
Selon les spécificités du projet, vous pouvez également inclure :
\begin{itemize}
    \item \textbf{Diagramme de composants :} pour visualiser la répartition modulaire du système et les dépendances entre les modules.
    \item \textbf{Diagramme d’activités :} pour représenter le flux de travail global ou des processus métier complexes.
    \item \textbf{Diagramme de déploiement :} pour illustrer la répartition des composants sur l’infrastructure matérielle (serveurs, conteneurs, cloud, etc.).
\end{itemize}

% =============================
% CONCLUSION
% =============================
\section{Conclusion}
Pour conclure ce chapitre, il convient de récapituler les éléments essentiels présentés :
\begin{itemize}
    \item \textbf{Synthèse des choix d’architecture et de modélisation :}  
    rappeler brièvement le modèle choisi (architecture et formalisme de modélisation) et les diagrammes 
    qui en résultent.
    
    \item \textbf{Justification des décisions :}  
    expliquer en quoi ces choix répondent aux besoins identifiés et aux contraintes du projet 
    (modularité, évolutivité, sécurité, performance, etc.).
    
    \item \textbf{Transition vers la phase de réalisation :}  
    indiquer que les modèles présentés serviront de base pour l’implémentation effective du système, 
    qui sera détaillée dans le chapitre suivant.
\end{itemize}

\noindent
Ce chapitre d'analyse et de conception constitue ainsi le socle technique sur lequel reposera la réalisation du projet.
