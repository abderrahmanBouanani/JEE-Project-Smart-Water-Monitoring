\chapter*{Abstract}
\addcontentsline{toc}{chapter}{Abstract}

Water scarcity and inefficient resource management pose significant challenges for sustainable urban development. This project presents Smart Water Monitoring, a comprehensive web platform for intelligent water consumption management using Internet of Things (IoT) technology and Java Enterprise Edition (JEE). The system addresses the critical need for real-time monitoring, automated data analysis, and proactive alert management to optimize water resource utilization in residential environments. The platform implements a multi-tier JEE architecture following the Model-View-Controller pattern, utilizing Jakarta Servlet API for request handling, Hibernate ORM for data persistence, and MySQL 8 for storage. Core functionalities include real-time IoT sensor data collection via REST APIs, automated daily consumption aggregation through scheduled tasks, intelligent alert generation based on configurable thresholds and time-based patterns, comprehensive statistical analysis, and role-based access control for administrators and citizens. Security features incorporate BCrypt password hashing with 12-round salting. A Python-based IoT simulator was developed to generate realistic water consumption data with hour-dependent patterns, enabling system validation and demonstration. The implementation successfully demonstrates a scalable, production-ready solution with automated background processing, real-time monitoring capabilities, and intuitive user interfaces. Performance evaluation shows efficient handling of multiple concurrent sensors with sub-second response times for API requests and reliable execution of scheduled aggregation jobs. This work contributes to smart city infrastructure by providing an open-source, modular platform for water resource management, with potential for future enhancements including machine learning-based predictive analytics and anomaly detection.

\medskip


\vspace{1cm}
\noindent
\textbf{Keywords:} Smart Water Monitoring, Internet of Things, Java Enterprise Edition, Water Resource Management, REST API

\vfill

\noindent
\textbf{Report's Total Word Count:} [To be completed after finalizing all chapters]

\medskip

\noindent
\textbf{GitHub Repository:} \url{https://github.com/abderrahmanBouanani/JEE-Project-Smart-Water-Monitoring}

\medskip
