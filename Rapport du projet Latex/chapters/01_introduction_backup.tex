\chapter*{\center{Introduction Générale}}
\addcontentsline{toc}{chapter}{Introduction Générale}

Cette section introductive propose une vision globale du projet et des travaux qui y sont associés. 
De manière générale, plusieurs éléments y sont abordés :

\begin{itemize}

    \item \textbf{Contexte et champ d’application} : un aperçu du cadre dans lequel s’inscrit le projet 
    (théories, systèmes, algorithmes, applications concrètes, etc.), permettant de bien cerner les enjeux 
    et la pertinence du travail à réaliser.
    
    \item \textbf{Description du problème} : une présentation succincte de la thématique ou de la question 
    étudiée, en soulignant l’importance du sujet et les difficultés potentielles.


    \item \textbf{Objectifs du projet} : la définition des buts visés — ce qui doit être accompli ou 
    démontré à la fin de l’étude ou du développement. Il peut s’agir de résoudre une problématique, 
    de concevoir un prototype, de réaliser une expérience, etc.

    \item \textbf{Approche et méthodologie} : la façon dont le problème sera traité ; il peut s’agir 
    de techniques d’implémentation, de protocoles expérimentaux ou de méthodes théoriques. 

    \item \textbf{Résultats et interprétations attendus} : un bref résumé des conclusions majeures 
    ou des retombées espérées (efficacité, performance, validation, etc.), sans toutefois rentrer 
    dans les détails.

    \item \textbf{Organisation du rapport} : un aperçu du contenu des chapitres à venir, afin de 
    guider le lecteur dans la structure globale du document.
\end{itemize}

\noindent
Il est fortement recommandé de \textbf{consulter votre encadrant} pour valider le contenu et 
l’étendue de cette introduction, car la forme et la longueur de celle-ci peuvent varier selon 
le type de projet (développement d’application, analyse algorithmique, travaux expérimentaux, 
approche théorique, etc.) et selon les préférences individuelles. 

Dans tous les cas, l’introduction doit fournir au lecteur \textbf{un cadre clair} sur la 
problématique, la démarche et les objectifs poursuivis, tout en donnant envie de découvrir 
la suite du rapport.
