\chapter{Technologies et réalisation}

\section{Introduction}

Ce chapitre présente la mise en œuvre du système \textbf{Smart Water Monitoring}. Nous détaillons les technologies utilisées, l'organisation du code, les interfaces développées et les fonctionnalités implémentées pour répondre aux besoins des citoyens et administrateurs.

L'objectif est de montrer comment les spécifications du projet ont été concrétisées en une application web fonctionnelle permettant la surveillance intelligente de la consommation d'eau.

\section{Choix techniques}

\subsection{Technologies utilisées}

\begin{itemize}
    \item \textbf{Backend} : Java EE avec Servlets et JSP
    \item \textbf{Base de données} : MySQL pour le stockage des données
    \item \textbf{ORM} : Hibernate pour la gestion de la persistance
    \item \textbf{Frontend} : HTML, CSS/Bootstrap, JavaScript avec JSP
    \item \textbf{Simulation} : Python pour les capteurs virtuels
\end{itemize}

\subsection{Outils de développement}

\begin{itemize}
    \item \textbf{Environnement} : IntelliJ IDEA pour Java, VS Code pour Python
    \item \textbf{Versioning} : Git avec GitHub
    \item \textbf{Base de données} : MySQL Workbench
    \item \textbf{Serveur} : Apache Tomcat
\end{itemize}

\section{Architecture du système}

Le système suit une architecture trois couches classique :

\begin{itemize}
    \item \textbf{Couche présentation} : Pages JSP pour l'interface utilisateur
    \item \textbf{Couche métier} : Servlets et services Java pour la logique applicative
    \item \textbf{Couche données} : Hibernate et MySQL pour la persistance
\end{itemize}

\section{Implémentation}

\subsection{Structure du projet}

Le code est organisé en packages selon les responsabilités :

\begin{verbatim}
src/
├── main/
│   ├── java/
│   │   ├── controller/          # Gestion des requêtes HTTP
│   │   ├── dao/                 # Accès aux données
│   │   ├── filter/              # Filtres de sécurité
│   │   ├── model/               # Entités métier
│   │   ├── services/            # Logique métier
│   │   └── jobs/                # Tâches planifiées
│   ├── resources/
│   │   └── META-INF/
│   │       └── persistence.xml  # Configuration base de données
│   └── webapp/
│       ├── WEB-INF/
│       │   └── web.xml          # Configuration déploiement
│       ├── views/               # Pages d'interface
│       ├── login.jsp
│       ├── signup.jsp
│       └── index.jsp
└── test/
    └── java/                    # Tests unitaires
\end{verbatim}
\subsection{Interfaces utilisateur}

\subsubsection{Interface citoyen}

L'interface citoyen offre les fonctionnalités suivantes :

\begin{itemize}
    \item Tableau de bord avec consommation en temps réel
    \item Visualisation des historiques et graphiques
    \item Consultation des alertes personnelles
    \item Profil utilisateur
\end{itemize}

\begin{figure}[H]
    \centering
    \includegraphics[width=0.8\textwidth]{figures/dashboard_citoyen.png}
    \caption{Tableau de bord citoyen}
    \label{fig:dashboard_citoyen}
\end{figure}

\subsubsection{Interface administrateur}

L'interface administrateur comprend :

\begin{itemize}
    \item Supervision globale du système
    \item Gestion des utilisateurs et capteurs
    \tableau de bord avec statistiques agrégées
    \item Rapports et analyses
\end{itemize}

\begin{figure}[H]
    \centering
    \includegraphics[width=0.8\textwidth]{figures/dashboard_admin.png}
    \caption{Tableau de bord administrateur}
    \label{fig:dashboard_admin}
\end{figure}

\subsection{Fonctionnalités principales}

\subsubsection{Côté client}

\begin{itemize}
    \item Authentification sécurisée
    \item Navigation intuitive selon le profil
    \item Affichage des données en temps réel
    \item Graphiques de consommation
    \item Gestion des formulaires
\end{itemize}

\subsubsection{Côté serveur}

\begin{itemize}
    \item API REST pour la collecte des données
    \item Gestion des sessions utilisateur
    \item Système de sécurité avec rôles
    \item Traitement et agrégation des données
    \item Génération automatique d'alertes
\end{itemize}

\subsubsection{Simulateur IoT}

\begin{itemize}
    \item Génération de données réalistes
    \item Envoi périodique vers le backend
    \item Gestion des patterns de consommation
    \item Configuration flexible
\end{itemize}

\section{Stratégie de validation}

Pour assurer la qualité du système, plusieurs types de tests ont été réalisés :

\subsection{Tests unitaires}

\begin{itemize}
    \item Validation de la logique métier
    \item Tests des services utilisateur et alertes
    \item Vérification des mots de passe
\end{itemize}

\subsection{Tests d'intégration}

\begin{itemize}
    \item Validation des workflows complets
    \item Tests de l'API de collecte de données
    \item Vérification de la persistance
\end{itemize}

\subsection{Tests manuels}

\begin{itemize}
    \item Validation des interfaces utilisateur
    \item Tests de navigation et ergonomie
    \item Vérification des scénarios métier
\end{itemize}

\section{Conclusion}

Ce chapitre a présenté la mise en œuvre technique du système Smart Water Monitoring. L'application web développée répond aux objectifs fixés en permettant la surveillance en temps réel de la consommation d'eau.

L'architecture en trois couches a structuré efficacement le développement, tandis que les technologies Java EE et MySQL ont fourni une base solide pour l'implémentation. Les interfaces utilisateur offrent une expérience adaptée aux différents profils, avec des tableaux de bord dédiés pour le suivi personnel et la gestion administrative.

Le système intègre avec succès les fonctionnalités principales : collecte de données via des capteurs virtuels, traitement automatique, génération d'alertes et visualisation des consommations. Les mécanismes de sécurité assurent la protection des données utilisateurs.

Cette réalisation démontre la faisabilité technique d'une solution complète de monitoring intelligent de l'eau.