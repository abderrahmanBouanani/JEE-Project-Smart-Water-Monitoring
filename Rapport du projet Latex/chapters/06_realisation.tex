\chapter{Technologies et réalisation}

Ce chapitre décrit l'ensemble des technologies, outils et méthodes qui ont été utilisés pour la mise en œuvre du projet. Il présente d'abord les choix techniques, puis détaille l'environnement de déploiement, l'implémentation du système (structure du code et interfaces utilisateur), ainsi que la stratégie de tests appliquée. Enfin, une conclusion synthétique fait le point sur les réalisations et les éventuelles difficultés rencontrées.

%%%%%%%%%%%%%%%%%%%%%%%%%%%%%%%%%%%%%%%%%%%%%%%%%%%%%%%%%%%
\section{Introduction}
%%%%%%%%%%%%%%%%%%%%%%%%%%%%%%%%%%%%%%%%%%%%%%%%%%%%%%%%%%%
Dans cette section introductive, l'objectif est de donner une vue d'ensemble des aspects techniques et pratiques abordés dans ce chapitre. On y explique notamment :
\begin{itemize}
    \item Les choix technologiques réalisés en fonction des besoins fonctionnels et des contraintes techniques.
    \item La manière dont l'environnement de déploiement est configuré pour supporter le système.
    \item Les détails de l'implémentation du code et de l'interface utilisateur.
    \item La stratégie adoptée pour valider et tester le système.
\end{itemize}
Cette partie prépare le lecteur à comprendre comment les décisions prises au niveau de la conception se traduisent concrètement lors de la réalisation.

%%%%%%%%%%%%%%%%%%%%%%%%%%%%%%%%%%%%%%%%%%%%%%%%%%%%%%%%%%%
\section{Choix techniques et outils}
%%%%%%%%%%%%%%%%%%%%%%%%%%%%%%%%%%%%%%%%%%%%%%%%%%%%%%%%%%%
Cette section détaille les principales technologies et outils qui ont été sélectionnés pour le développement du projet.

\subsection{Langages, Frameworks, Bases de données…}
\begin{itemize}
    \item \textbf{Langages de programmation :}  
    Justifier le choix du ou des langages (par exemple, Java pour la robustesse et la scalabilité, Python pour la rapidité de développement ou JavaScript/React pour le front-end interactif).
    \item \textbf{Frameworks :}  
    Expliquer la sélection des frameworks, tels que Spring Boot pour le back-end, React ou Angular pour le front-end, ou encore d'autres frameworks spécifiques à la tâche.
    \item \textbf{Bases de données :}  
    Préciser le type de base de données utilisée (relationnelle comme MySQL/PostgreSQL ou NoSQL comme MongoDB) en fonction des critères de performance, de flexibilité et de volume de données.
    \item \textbf{Critères de sélection :}  
    Indiquer les critères qui ont guidé ces choix, par exemple la taille de la communauté, la robustesse, la simplicité d'intégration et la performance.
\end{itemize}
Exemple d'utilisation des références : 
Dans ce rapport, diverses références ont été utilisées pour étayer la démarche adoptée. Par exemple, la conception de sites web modernes est détaillée dans \cite{duckett2011html}, tandis que \cite{horstmann2014jee} offre une vue d'ensemble de l'architecture Java EE. Pour une approche approfondie de l'ORM avec Hibernate, on peut se référer à \cite{bauer2007hibernate}. Enfin, \cite{keogh2002jsp} fournit un guide complet sur le développement de JavaServer Pages.

\subsection{Outils de développement (IDE, gestion de versions…)}
\begin{itemize}
    \item \textbf{Environnement de développement intégré (IDE) :}  
    Mentionner les outils utilisés, par exemple VS Code, Eclipse ou IntelliJ IDEA, et expliquer pourquoi ces outils ont été privilégiés.
    \item \textbf{Système de gestion de versions :}  
    Indiquer l'utilisation d'un outil comme Git (hébergé sur GitHub ou GitLab) pour assurer le suivi des modifications, faciliter la collaboration et la gestion du code source.
    \item \textbf{Autres outils de collaboration et de tests :}  
    Par exemple, l'utilisation de Maven ou Gradle pour la gestion de projet, ainsi que des outils de CI/CD pour automatiser les builds et les tests.
\end{itemize}

%%%%%%%%%%%%%%%%%%%%%%%%%%%%%%%%%%%%%%%%%%%%%%%%%%%%%%%%%%%
\section{Architecture de déploiement}
%%%%%%%%%%%%%%%%%%%%%%%%%%%%%%%%%%%%%%%%%%%%%%%%%%%%%%%%%%%
Cette section décrit l'infrastructure sur laquelle le système sera déployé et les outils utilisés pour assurer son fonctionnement continu.

\begin{itemize}
    \item \textbf{Environnement cible :}  
    Décrire le ou les serveurs, l'utilisation éventuelle de conteneurs (Docker) ou du cloud (AWS, Azure, Google Cloud), ainsi que les systèmes d'exploitation concernés.
    \item \textbf{Pipeline d'intégration continue (CI/CD) :}  
    Indiquer si un pipeline CI/CD a été mis en place pour automatiser les phases de build, de tests et de déploiement, et présenter les outils utilisés (Jenkins, GitLab CI, Travis CI, etc.).
\end{itemize}

%%%%%%%%%%%%%%%%%%%%%%%%%%%%%%%%%%%%%%%%%%%%%%%%%%%%%%%%%%%
\section{Implémentation et interfaces}
%%%%%%%%%%%%%%%%%%%%%%%%%%%%%%%%%%%%%%%%%%%%%%%%%%%%%%%%%%%
Cette section est consacrée à la description de l'implémentation du système ainsi qu'à la présentation des interfaces utilisateur.

\begin{itemize}
    \item \textbf{Structure du code :}  
    Exposer l'organisation du code en packages ou modules. Décrire brièvement la répartition entre le front-end, la logique métier et la couche de données.
    \item \textbf{Interfaces utilisateur :}  
    Présenter l’interface graphique à travers des captures d’écran ou des maquettes. Décrire les principales fonctionnalités disponibles pour l’utilisateur final.
    \item \textbf{Fonctionnalités principales :}  
    Détaillez les fonctionnalités clés implémentées, tant côté front-end (navigation, formulaires, affichage des données) que côté back-end (gestion des requêtes, traitement des données, sécurité).
\end{itemize}

%%%%%%%%%%%%%%%%%%%%%%%%%%%%%%%%%%%%%%%%%%%%%%%%%%%%%%%%%%%
\section{Stratégie de tests}
%%%%%%%%%%%%%%%%%%%%%%%%%%%%%%%%%%%%%%%%%%%%%%%%%%%%%%%%%%%
Pour garantir la qualité et la robustesse du système, une stratégie de tests complète a été mise en place.

\begin{itemize}
    \item \textbf{Tests unitaires :}  
    Décrire la couverture des tests unitaires qui vérifient le fonctionnement de chaque composant individuel (par exemple, avec JUnit, pytest, etc.).
    \item \textbf{Tests d'intégration :}  
    Expliquer comment les différents modules du système ont été testés ensemble pour s'assurer de la bonne communication entre eux.
    \item \textbf{Tests système :}  
    Préciser si des tests globaux ont été effectués pour valider le comportement complet du système, incluant l’interface utilisateur.
    \item \textbf{Outils de test et plan de validation :}  
    Mentionner les outils utilisés (Selenium pour les tests d’interface, Postman pour les API, etc.) et résumer le plan de validation ainsi que les résultats obtenus de manière synthétique.
\end{itemize}

%%%%%%%%%%%%%%%%%%%%%%%%%%%%%%%%%%%%%%%%%%%%%%%%%%%%%%%%%%%
\section{Conclusion}
%%%%%%%%%%%%%%%%%%%%%%%%%%%%%%%%%%%%%%%%%%%%%%%%%%%%%%%%%%%
Pour clore ce chapitre, il est important de faire un bilan sur l’ensemble des décisions techniques et 
de réalisation qui ont été prises :

\begin{itemize}
    \item \textbf{Synthèse des choix techniques :}  
    Récapituler les technologies, outils et méthodologies adoptés, en insistant sur leur adéquation 
    avec les besoins du projet (simplicité d’utilisation, performance, évolutivité, etc.).
    \item \textbf{Retour sur l'architecture de déploiement et l'implémentation :}  
    Expliquer brièvement comment l'environnement de déploiement et la structure du code assurent la 
    stabilité et la pérennité du système.
    \item \textbf{Perspectives et défis :}  
    Mentionner les difficultés éventuelles rencontrées lors de l’implémentation (limitations techniques, 
    problèmes d’intégration, etc.) et indiquer les pistes d’amélioration à envisager pour les phases ultérieures.
    \item \textbf{Transition vers le chapitre suivant :}  
    Annoncer que les éléments techniques présentés ici constitueront la base pour l’implémentation 
    concrète du système, qui sera détaillée dans le chapitre de réalisation pratique.
\end{itemize}

\noindent
Ce chapitre constitue ainsi un socle technique solide, garantissant que les choix de développement 
sont cohérents avec les objectifs du projet et prêts à être mis en œuvre lors de la phase de réalisation.
