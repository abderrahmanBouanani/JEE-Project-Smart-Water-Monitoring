\chapter*{ \center Conclusion générale}
\addcontentsline{toc}{chapter}{Conclusion générale}

Ce projet de \textbf{Smart Water Monitoring System} a permis de développer une plateforme web complète pour la surveillance intelligente de la consommation d'eau.

\section*{Objectifs atteints}

Les objectifs initiaux ont été remplis avec succès :
\begin{itemize}
    \item Développement d'une application web fonctionnelle avec Java EE
    \item Collecte et traitement des données de consommation en temps réel
    \item Création d'interfaces adaptées pour les citoyens et administrateurs
    \item Mise en place d'un système d'alertes automatiques
    \item Simulation réaliste de capteurs IoT avec Python
\end{itemize}

\section*{Réalisations principales}

Le système offre des fonctionnalités concrètes :
\begin{itemize}
    \item Tableaux de bord interactifs pour le suivi de la consommation
    \item Détection automatique des fuites et surconsommations
    \item Gestion sécurisée des comptes utilisateurs
    \item Agrégation quotidienne des données
    \item Visualisation des historiques et statistiques
\end{itemize}

\section*{Limites identifiées}

Quelques limitations sont à noter :
\begin{itemize}
    \item Utilisation de capteurs virtuels plutôt que physiques
    \item Interface basée sur JSP sans framework moderne
    \item Tests limités par les contraintes académiques
\end{itemize}

\section*{Perspectives futures}

Des améliorations pourraient être apportées :
\begin{itemize}
    \item Intégration de vrais capteurs IoT
    \item Développement d'une application mobile
    \item Ajout de fonctionnalités d'analyse avancée
    \item Modernisation de l'interface utilisateur
\end{itemize}

Ce projet démontre la faisabilité d'un système de monitoring intelligent de l'eau et constitue une base solide pour des développements futurs dans le domaine des villes intelligentes et de la gestion durable des ressources.