\chapter*{ \center Conclusion générale}
\addcontentsline{toc}{chapter}{Conclusion générale}

La conclusion générale met en lumière les principaux enseignements et aboutissements du projet. Elle récapitule les objectifs initiaux, synthétise les contributions et résultats obtenus, identifie les limites du travail réalisé, et propose des pistes pour des développements futurs. 

\begin{itemize}
    \item \textbf{Récapitulation des objectifs initiaux :}
    \begin{itemize}
        \item Les problématiques et questions de recherche abordées dès le départ sont rappelées afin de souligner l'enjeu initial du projet.
        \item Une évaluation est présentée pour vérifier dans quelle mesure les objectifs techniques et fonctionnels ont été atteints.
    \end{itemize}
    
    \item \textbf{Contributions et résultats obtenus :}
    \begin{itemize}
        \item Les réalisations concrètes du projet (fonctionnalités développées, prototypes validés, analyses réalisées, etc.) sont synthétisées.
        \item Les apports spécifiques du projet sont mis en avant, qu'il s'agisse d'innovations techniques, de gains de temps ou d'améliorations de performance.
    \end{itemize}
    
    \item \textbf{Limites du travail :}
    \begin{itemize}
        \item Les points de blocage et contraintes rencontrés (budgétaires, délais, aspects techniques) sont identifiés.
        \item Une analyse critique montre en quoi ces limites peuvent influencer l'utilisation ou l'évolution du système développé.
    \end{itemize}
    
    \item \textbf{Perspectives et améliorations futures :}
    \begin{itemize}
        \item Des pistes d'évolution sont proposées, telles que l'ajout de nouvelles fonctionnalités, l'optimisation des performances, ou la migration vers des technologies plus avancées.
        \item Pour les projets à dimension scientifique, des axes de recherche complémentaires sont suggérés pour approfondir les résultats obtenus.
    \end{itemize}
\end{itemize}

