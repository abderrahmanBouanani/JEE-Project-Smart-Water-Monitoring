\chapter{Contexte général du projet}

\section{Introduction}
Ce chapitre a pour vocation de situer le projet dans son ensemble et de préciser 
les différents paramètres qui en encadrent la réalisation. Il permet de comprendre 
le lien avec les développements antérieurs (présentés dans le chapitre précédent, 
le cas échéant) ainsi que la façon dont il s’inscrit dans la continuité du rapport. 
Plus particulièrement:

\begin{itemize}
    \item \textbf{Rappel du contexte :} 
    Il est essentiel de rappeler les grandes lignes du domaine ou de la problématique 
    déjà abordées auparavant. Cela permet de maintenir une cohérence globale et de souligner 
    la progression logique du rapport.
    
    \item \textbf{Objectif du chapitre :} 
    Ce chapitre vise à décrire la nature du projet, sa finalité ainsi que le champ de 
    recherches ou d’applications qu’il couvre. Les informations présentées orientent et 
    justifient les choix à venir. Elles sont également nécessaires pour comprendre les 
    motivations et la portée du travail réalisé.
\end{itemize}

\section{Présentation du projet}
\subsection{Sujet du projet}
Le \textbf{choix du sujet} résulte souvent d’une commande (entreprise, institution, 
cahier des charges) ou d’une problématique identifiée (besoin d’automatisation, 
d’amélioration de processus, etc.). Les points suivants méritent d’être clarifiés :

\begin{itemize}
    \item \textbf{Origine du projet :} 
    Le contexte (académique, professionnel, industriel) dans lequel l’idée a émergé, 
    ainsi que les motivations (ex. résolution d’un problème concret, innovation technique…).
    
    \item \textbf{Domaine d’application et utilisateurs cibles :} 
    Identifier qui seront les principaux bénéficiaires ou acteurs (étudiants, professeurs, 
    clients, opérateurs, etc.). Il est important de préciser les besoins et contraintes 
    propres à ce public (environnement technique, niveau d’expertise, etc.).
\end{itemize}

\subsection{Intérêt du projet}
Cette partie met en avant la \textbf{plus-value attendue} du projet :

\begin{itemize}
    \item \textbf{Améliorations apportées :} 
    Souligner en quoi le projet se démarque de l’existant (par exemple, automatisation 
    d’une tâche auparavant manuelle, optimisation d’une application, gain de temps…).
    
    \item \textbf{Impact et pertinence :} 
    Expliquer pourquoi il est important ou nécessaire de résoudre cette problématique. 
    Dans quel cas d’usage ou contexte (universitaire, industriel, etc.) cette solution 
    présente-t-elle un véritable atout ?
\end{itemize}

\section{Problématique et état de l’existant}
Afin de \textbf{justifier} la mise en place du projet, il est souvent crucial de procéder 
à une analyse de la littérature ou des solutions déjà disponibles :

\begin{itemize}
    \item \textbf{Solutions ou méthodes proposées :} 
    Cette étape peut consister en une revue de l’existant (logiciels, algorithmes, 
    démarches) ou en l’observation d’un processus manuel devenu obsolète (par exemple, 
    saisie manuelle fastidieuse). Les principales forces et faiblesses de ces approches 
    doivent être soulignées.
    
    \item \textbf{Défauts et limites :} 
    Souligner clairement les points qui ne sont pas (ou mal) couverts par les solutions 
    actuelles : complexité excessive, coûts trop élevés, manque d’ergonomie, etc. 
    Ces lacunes mettent en évidence la \textbf{nécessité} d’une nouvelle approche 
    ou d’une amélioration.
\end{itemize}

\section{Objectifs du projet}
Les objectifs offrent un cadre structurant pour l’ensemble du travail. Ils peuvent être 
\textbf{classés en différentes catégories} :

\begin{itemize}
    \item \textbf{Objectifs fonctionnels :} 
    Il s’agit de décrire les fonctionnalités attendues du système ou de l’application. 
    Par exemple : gérer des comptes utilisateurs, effectuer un calcul d’angles, permettre 
    la visualisation de données en temps réel, etc.
    
    \item \textbf{Objectifs techniques :} 
    On aborde ici les aspects liés à la performance (temps de réponse, robustesse du système), 
    à la maintenabilité (structure modulaire, documentation), ou encore aux technologies 
    privilégiées (langages, frameworks, bases de données…). Ces objectifs peuvent inclure 
    des considérations de sécurité ou de conformité à des standards.
    
    \item \textbf{Contraintes :} 
    Les contraintes peuvent être multiples : 
    \begin{itemize}
        \item \textbf{Temporelles :} date butoir pour la livraison, temps limité pour 
        la phase de développement ou de test.
        \item \textbf{Budgétaires :} moyens financiers alloués ou ressources matérielles 
        disponibles.
        \item \textbf{Organisationnelles :} disponibilité des intervenants, politiques 
        internes de l’institution, etc.
    \end{itemize}
\end{itemize}

\section{Démarche de gestion de projet}
\subsection{Méthodologie (Scrum, Cycle en V, etc.)}
Cette partie explicite la \textbf{méthode de gestion de projet} choisie et la 
façon dont elle s’adapte au contexte :

\begin{itemize}
    \item \textbf{Présenter la méthode :} 
    Par exemple, Scrum (méthode agile) favorise l’adaptabilité et la communication 
    fréquente avec le client, tandis que le Cycle en V implique des étapes plus linéaires 
    (spécification, conception, validation).
    
    \item \textbf{Rôle des intervenants :} 
    Définir qui est le client (ou commanditaire), le product owner, l’équipe de développement, 
    etc., et expliquer brièvement les interactions (séances de feedback, revues intermédiaires…).
\end{itemize}


    La figure \ref{fig:scrum} présnete le schéma du cycle Scrum et de ses principales étapes.
    \begin{figure}[ht]
        \centering
        \includegraphics[width=0.9\linewidth]{figures/scrum.png}
        \caption{Schéma du cycle Scrum et de ses principales étapes.}
        \label{fig:scrum}
    \end{figure}

\subsection{Planification du projet}
La planification vient concrétiser la méthode sélectionnée :

\begin{itemize}
    \item \textbf{Phases ou sprints :} 
    Décrire chaque phase de manière synthétique : objectifs, durée, livrables à fournir. 
    Si vous utilisez Scrum, mentionnez le nombre de sprints et leur contenu. 
    Dans le cas d’un Cycle en V, mettez l’accent sur les étapes (spécification, conception, 
    implémentation, tests).
    
    \item \textbf{Outils de planification :} 
    Un diagramme de Gantt peut illustrer le calendrier global, tandis qu’un backlog Scrum 
    répertorie l’ensemble des tâches à accomplir. L’idéal est de décrire comment vous assurez 
    le suivi (réunions quotidiennes, rétrospectives, logiciel de gestion de tâches, etc.).
\end{itemize}

Le tableau \ref{tab:sprints} présente le planning détaillé des sprints du projet, incluant les objectifs spécifiques, la durée de chaque sprint ainsi que les livrables attendus à l'issue de chaque phase.

\begin{table}[ht]
\centering
\caption{Planification des sprints.}
\label{tab:sprints}
\begin{tabular}{|c|p{7cm}|c|p{4cm}|}
\hline
\textbf{Sprint} & \textbf{Objectifs} & \textbf{Durée} & \textbf{Livrables} \\ \hline
1 & Mise en place de l'environnement de développement, configuration des outils et préparation du dépôt de code. & 2 semaines & Environnement configuré, dépôt initial \\ \hline
2 & Développement du module d'authentification et gestion des utilisateurs. & 3 semaines & Module d'authentification opérationnel, tests unitaires \\ \hline
3 & Conception de l'interface utilisateur et début de la gestion des données. & 2 semaines & Maquettes validées, premiers écrans fonctionnels \\ \hline
4 & Intégration des fonctionnalités de base et réalisation de tests d'intégration. & 3 semaines & Application intégrée, rapport de tests \\ \hline
5 & Optimisation des performances et finalisation des modules, préparation de la version finale. & 2 semaines & Version finale livrée, documentation complète \\ \hline
\end{tabular}
\end{table}


\section{Conclusion}
En guise de synthèse, il est recommandé de récapituler les éléments clés présentés, 
afin de poser des bases solides pour la suite :

\begin{itemize}
    \item \textbf{Synthèse du contenu :} 
    Le chapitre a permis de dresser un portrait global du projet : on connaît désormais 
    ses motivations, les solutions existantes, les objectifs à atteindre et les contraintes 
    à respecter. Ces informations permettent de comprendre la pertinence et la finalité du travail.
    
    \item \textbf{Limites et difficultés rencontrées :} 
    Si certaines incertitudes ou obstacles ont été identifiés (par exemple, un manque de clarté 
    dans le cahier des charges, des variables non maîtrisées, etc.), il convient de les mentionner. 
    On peut esquisser des pistes de solutions ou décider de les aborder dans un chapitre ultérieur.
    
    \item \textbf{Transition vers le chapitre suivant :} 
    Les informations collectées et analysées ici constitueront la base de l’étude fonctionnelle 
    et/ou de la conception à venir. Le prochain chapitre pourra, par exemple, détailler 
    l’architecture envisagée, le modèle de données ou encore les diagrammes UML. 
    Cette transition assure la cohérence de la démarche globale et guide le lecteur dans la 
    progression du rapport.
\end{itemize}
