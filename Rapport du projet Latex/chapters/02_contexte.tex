\chapter{Contexte général du projet}

\section{Introduction}
Ce chapitre a pour vocation de situer le projet dans son ensemble et de préciser les différents paramètres qui en encadrent la réalisation. Il permet de comprendre le lien avec les développements antérieurs et la façon dont il s’inscrit dans la continuité du rapport. Plus particulièrement, ce chapitre vise à décrire la nature du projet, sa finalité ainsi que le champ de recherches ou d’applications qu’il couvre. Les informations présentées orientent et justifient les choix à venir, et sont également nécessaires pour comprendre les motivations et la portée du travail réalisé.

\section{Présentation du projet}
\subsection{Sujet du projet}
Le projet \textbf{Smart Water Monitoring System} est né d'un besoin croissant de moderniser la gestion des ressources en eau face aux défis climatiques et démographiques actuels. Il s'inscrit dans un contexte académique au sein de l'ENSA Agadir, visant à appliquer les concepts du module JEE à une problématique concrète et d'actualité. L'origine du projet est une initiative visant à explorer comment les technologies de l'Internet des Objets (IoT) et les plateformes web peuvent contribuer à une gestion plus efficiente et durable de l'eau.

Le domaine d'application est celui de la \textit{Smart City} (ville intelligente), où la technologie est mise au service de l'amélioration des services urbains. Les utilisateurs cibles sont multiples :
\begin{itemize}
    \item \textbf{Les citoyens :} qui pourront suivre leur consommation en temps réel, recevoir des alertes en cas de fuite, et adopter des comportements plus responsables.
    \item \textbf{Les administrateurs du réseau d'eau :} qui disposeront d'un outil de supervision centralisé pour surveiller l'état du réseau, analyser les données de consommation et optimiser la distribution.
\end{itemize}

\subsection{Intérêt du projet}
La plus-value attendue de ce projet est significative. Il vise à remplacer les méthodes traditionnelles de relevé de compteurs, souvent manuelles et peu fréquentes, par un système automatisé et en temps réel \cite{zhang2023wateriot, gupta2021smartwater}. Le tableau \ref{tab:comparaison_methodes} met en évidence les améliorations apportées par rapport à l'existant.

\begin{table}[ht]
\centering
\caption{Comparaison entre la gestion traditionnelle et la solution proposée.}
\label{tab:comparaison_methodes}
\begin{tabular}{|l|p{5cm}|p{5cm}|}
\hline
\textbf{Critère} & \textbf{Gestion traditionnelle} & \textbf{Solution Smart Water Monitoring} \\ \hline
Fréquence des relevés & Manuelle (mensuelle/annuelle) & Automatisée et en temps réel \\ \hline
Détection des fuites & Lente et souvent tardive & Instantanée avec alertes automatiques \\ \hline
Visibilité pour l'utilisateur & Faible (factures périodiques) & Élevée (tableaux de bord interactifs) \\ \hline
Analyse des données & Limitée, données agrégées & Approfondie, tendances et prévisions \\ \hline
Optimisation & Difficile, manque d'informations & Facilitée par des objectifs personnalisés \\ \hline
\end{tabular}
\end{table}

L'impact du projet est double : il permet non seulement de réaliser des économies d'eau et de réduire le gaspillage, mais aussi de sensibiliser les consommateurs à l'importance de préserver cette ressource \cite{iea2022smartcities}. Dans un contexte industriel, une telle solution pourrait être déployée à grande échelle pour optimiser la gestion des réseaux de distribution d'eau potable \cite{martin2021waterleaks}.

\section{Problématique et état de l'existant}
Pour justifier la pertinence du projet, une analyse des solutions existantes a été menée \cite{gupta2021smartwater, silva2022iotwater}. De nombreuses plateformes de gestion de l'eau existent, mais elles présentent souvent des limites \cite{zhang2023wateriot}. Certaines sont des systèmes propriétaires coûteux et peu flexibles, tandis que d'autres sont des solutions open-source complexes à déployer et à maintenir \cite{iea2022smartcities}.

Les principales faiblesses des approches actuelles sont \cite{silva2022iotwater, martin2021waterleaks} :
\begin{itemize}
    \item \textbf{Complexité d'intégration :} Difficulté à intégrer des capteurs de différents fabricants.
    \item \textbf{Coûts élevés :} Licences logicielles et matériel propriétaire onéreux.
    \item \textbf{Manque de personnalisation :} Interfaces et fonctionnalités non adaptées aux besoins spécifiques des utilisateurs.
\end{itemize}
Ces lacunes soulignent la nécessité de développer une solution modulaire, open-source et basée sur des technologies standard comme Jakarta EE, offrant ainsi une alternative flexible et économique \cite{horstmann2014jee}.

\section{Objectifs du projet}
Les objectifs du projet se déclinent en plusieurs catégories.

\subsection{Objectifs fonctionnels}
\begin{itemize}
    \item Gérer les comptes utilisateurs (citoyens, administrateurs) avec authentification sécurisée.
    \item Collecter et stocker les données de consommation provenant des capteurs IoT.
    \item Afficher la consommation d'eau en temps réel et l'historique via des graphiques.
    \item Permettre aux utilisateurs de définir des objectifs de consommation mensuels.
    \item Générer des alertes automatiques en cas de détection de fuites ou de surconsommation.
    \item Fournir des statistiques agrégées (consommation moyenne, pics, etc.) aux administrateurs.
\end{itemize}

\subsection{Objectifs techniques}
\begin{itemize}
    \item Développer une application web robuste et scalable en utilisant Jakarta Servlet et Hibernate.
    \item Assurer la sécurité des données par le hachage des mots de passe (BCrypt) et la gestion des sessions.
    \item Mettre en place une communication fiable entre le simulateur IoT (Python) et le backend Java via une API REST.
    \item Concevoir une base de données relationnelle optimisée avec MySQL 8.
    \item Garantir un temps de réponse rapide pour l'affichage des données en temps réel.
\end{itemize}

\subsection{Contraintes}
Le projet est soumis à plusieurs contraintes, notamment :
\begin{itemize}
    \item \textbf{Temporelles :} Le développement doit être réalisé dans le cadre du semestre académique.
    \item \textbf{Technologiques :} L'utilisation de Jakarta EE, Hibernate et MySQL est imposée par le cahier des charges du module.
    \item \textbf{Matérielles :} Le projet ne dispose pas de capteurs IoT physiques, ce qui a nécessité le développement d'un simulateur.
\end{itemize}

\section{Démarche de gestion de projet}
\subsection{Méthodologie (Scrum)}
Pour la gestion de ce projet, nous avons adopté la méthodologie agile \textbf{Scrum} \cite{schwaber2020scrum}. Cette méthode favorise l'adaptabilité, la collaboration et la livraison itérative de fonctionnalités. Elle nous a permis de découper le projet en sprints, de prioriser les tâches et de nous adapter rapidement aux défis techniques rencontrés. L'équipe de développement est composée des deux étudiants, qui interagissent régulièrement avec l'enseignant (jouant le rôle de Product Owner) pour valider les fonctionnalités développées. La figure \ref{fig:scrum} présente le schéma du cycle Scrum et de ses principales étapes.

\begin{figure}[ht]
    \centering
    \includegraphics[width=0.9\linewidth]{figures/scrum.png}
    \caption{Schéma du cycle Scrum et de ses principales étapes.}
    \label{fig:scrum}
\end{figure}

\subsection{Planification du projet}
La planification du projet a été organisée en cinq sprints, chacun avec des objectifs clairs et des livrables spécifiques. Le tableau \ref{tab:sprints} montre le planning détaillé des sprints du projet.

\begin{table}[H]
\centering
\caption{Planification des sprints du projet Smart Water Monitoring.}
\label{tab:sprints}
\footnotesize
\begin{tabular}{|c|p{5cm}|c|p{3.5cm}|}
\hline
\textbf{Sprint} & \textbf{Objectifs} & \textbf{Durée} & \textbf{Livrables} \\ \hline
1 & Mise en place de l'environnement, initialisation du projet et conception de la base de données. & 2 semaines & Environnement configuré, dépôt Git, schéma de la BDD \\ \hline
2 & Développement du module d'authentification (signup, login, logout) et gestion des utilisateurs. & 3 semaines & Module d'authentification fonctionnel, tests unitaires \\ \hline
3 & Développement du simulateur IoT et de l'API REST pour la collecte des données de consommation. & 2 semaines & Simulateur Python, Endpoints REST pour les données \\ \hline
4 & Conception des tableaux de bord (citoyen et admin) et affichage des données en temps réel. & 3 semaines & Interfaces de visualisation des données \\ \hline
5 & Implémentation du système d'alertes, des objectifs de consommation et finalisation. & 2 semaines & Version finale de l'application, documentation \\ \hline
\end{tabular}
\end{table}

Le suivi du projet a été assuré par des réunions hebdomadaires et l'utilisation d'un tableau de bord de type Kanban pour visualiser l'avancement des tâches.

\section{Conclusion}
Ce chapitre a permis de définir le cadre du projet \textbf{Smart Water Monitoring System}. Nous avons présenté ses motivations, analysé les solutions existantes, et défini des objectifs clairs et mesurables. La démarche de gestion de projet basée sur Scrum nous a fourni un cadre de travail structuré et flexible. Les informations collectées ici constituent la base sur laquelle nous allons construire l'analyse et la conception détaillées dans les chapitres suivants. Le prochain chapitre se concentrera sur l'étude préliminaire des technologies clés du projet.
